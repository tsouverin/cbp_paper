\section{Introduction}

The calibration of optical wide-field surveys needs to reach new levels of precision to meet the requirements of type Ia supernovae (SNe Ia) cosmology. SNe Ia are standard candles, a class of objects with predictable luminosity used as probes to characterize dark energy in the late Universe. We can infer dark energy properties by measuring the luminosity distance of SNe Ia at different redshifts. This luminosity distance is obtained by measuring the maximum amplitude of the SN Ia light curve, which is observed within different optical bands depending on its redshift. Errors in the relative flux calibration between the different bands have a knock-on effect on systematic errors in the Hubble diagram, which are then propagated to dark energy parameters constraints.

Current photometric surveys like the Dark Energy Survey (DES) \citep{Brout_2019} or Subaru Hyper Suprime-Cam (HSC) \citep{hsc_2019} observed hundreds of SNe Ia. The Zwicky Transient Facility (ZTF) \citep{ztf_2022} should total up to about \num{10000} spectroscopically confirmed SNe Ia, a consequent increase compared to previous surveys. Joint analysis has been performed to benefit from the different existing surveys, reducing the statistical uncertainty on the measurement of cosmological parameters (\citealt{Betoule_2014,Scolnic_2018,Brout_2022,rubin2023union}). Additionally, we expect the SNe Ia catalog to reach a new order of magnitude within the next decade thanks to the Legacy Survey of Space and Time (LSST) undertaken by the Vera Rubin Observatory \citep{lsst}, which is expected to observe between 120,000 and 170,000 SNe Ia up to redshifts $z \sim 0.3$ \citep{lsst_2022}. With this tremendous increase, the statistical uncertainty on cosmological parameters will consequently diminish, leaving the photometric calibration as one primary source of systematic uncertainty. Therefore, the photometric calibration needs to reach sub-percent precision to benefit from the incoming statistics of the present and future surveys.

SNe Ia survey bandpasses are calibrated relatively to the CALSPEC catalog of spectrophotometric standard stars (\cite{Bohlin_2020}), which relies on the radiative transmission model of white dwarf atmospheres (\cite{Narayan_2019}). To transfer this calibration to the photometric survey, it is necessary to consider two additional components: (i) the terrestrial atmospheric transmission and (ii) the survey filter transmissions as a function of wavelength, with particular attention given to the bandpasses edges wavelength position. The former can be inferred by airmass regression with observations of CALSPEC reference stars or by slitless spectrophotometric analysis with a dedicated telescope such as the Rubin Observatory's auxiliary telescope (AuxTel), as detailed in \cite{Neveu_2024}. The latter needs beforehand precise measurements of the bandpasses throughput. Multiple strategies have been developed to provide this measurement on different surveys. The most common approach consists of using calibrated sensors such as the ones supplied by the National Institute of Standards and Technologies (NIST) \citep{houston2008detectors} to monitor a light source used to illuminate a telescope to measure its throughput and filter transmissions. Several approaches involve diffusion on a flat-field screen toward the instrument \citep{stubbs2006,marshall2013}. Other designs have been developed, like \cite{Lombardo_2017}, which involves integrating spheres and parabolic mirrors to redirect the light in a parallel beam.

The StarDICE experiment (\citealt{Betoule_2023}) proposes a metrology chain from laboratory flux references toward the measurement of standard star spectra. Several steps are needed to calibrate increasingly sensitive detectors and finally transfer this calibration to on-sky sources. The StarDICE \SI{40}{\centi\meter} diameter telescope is calibrated with a stable light source positioned far enough (\textasciitilde {\SI{100}{\meter}) to appear as a pointlike source and provides in-situ calibration of the instrument. Composed of LEDs, the calibrated light will emit broadband flux, which will be used to monitor the $ugrizy$ filters of the \SD{} telescope at the millimagnitude level. Beforehand, it is necessary to have a laboratory measurement of the filter transmission at high wavelength resolution to interpret the broadband LED and star measurements. In this context, a Collimated Beam Projector (hereafter CBP) has been developed to accurately measure the StarDICE telescope response $\Rtel(\lambda)$, including the optics, the CCD camera quantum efficiency, and the filter transmission. 

The CBP was initially designed for the LSST telescope (\citealt{ingraham2016}), and prototypes have been used to measure the throughput of the CTIO \SI{0.9}{\meter} telescope and the DECam wide field imager, respectively, in \cite{coughlin2018} and \cite{coughlin2016}. Compared to flat-field illumination devices, the CBP can project monochromatic light in a collimated beam, which provides parallel monochromatic illumination over a portion of the primary mirror. A first prototype of the CBP for StarDICE has been developed in \cite{Mondrik_2023} as a proof-of-concept and measured the instrument throughput with a precision of \textasciitilde 3\% for wavelengths between \SI{400}{\nano\meter} and \SI{800}{\nano\meter}, and a wavelength calibration estimated at \textasciitilde \SI{0.2}{\nano\meter}. 

This paper details the enhanced version of the StarDICE CBP, now equipped with a tunable laser as a monochromatic source that injects light at the focal point of a Ritchey-Chrétien telescope mounted backward. The collimated CBP output beam illuminates the StarDICE telescope pupil to measure its throughput. Incorporating lessons from previous CBP prototypes, and leveraging calibration efforts provided by \cite{houston2008detectors} and \cite{solarcell} for the detectors, our goal is to achieve sub-percent precision in measuring the StarDICE telescope throughput and $ugrizy$ filter transmissions while confirming the \SI{0.2}{\nano\meter} wavelength calibration accuracy. This study has two major intents: (i) providing a first measurement of the StarDICE filter transmission to contribute to the StarDICE experiment's metrology chain, and (ii) serving as a pathfinder for the future measurement of the LSST telescope with its dedicated version of the CBP.

The study follows three main steps. First, we illuminate a calibrated solar cell with the CBP to determine its throughput, using a wavelength sampling of \SI{1}{nm} across the 350 to \SI{1100}{nm} range. Second, we direct the calibrated CBP beam toward the StarDICE telescope to measure its throughput across the same wavelength range. Since the CBP output does not fully illuminate the StarDICE primary mirror, we perform a series of measurements at different positions, repeating this step for each position. Finally, we reconstruct the telescope's full-pupil response by interpolating the measurements across the scanned positions. Several factors must be carefully controlled to ensure accurate CBP throughput calibration, including laser spectral purity, minimizing scattered light (which differently affects the solar cell and the telescope), and achieving a sufficient signal-to-noise ratio when measuring the CBP output beam, given the high $1/f$ noise (pink noise) in these detectors.

The remainder of this paper details the acquisition plan, hardware setup, and software tools used to achieve a sub-percent calibration of the StarDICE telescope response and filter transmissions. The structure is as follows: Section~\ref{sec:setup} describes the laboratory setup and measurement campaign. Section~\ref{sec:rcbp} details CBP optical response measurements. Section~\ref{sec:rsd} presents the measured StarDICE telescope throughput and filter transmissions. Section~\ref{sec:pupil_stitching} discusses the methodology for synthesizing the full-pupil response, with results and implications discussed in Section~\ref{sec:discussion}.
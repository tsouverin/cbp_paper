\section{Conclusion}
\label{sec:discussion}

We determined the response of the 16'' StarDICE telescope using a
collimated beam projector (CBP) powered by a tunable laser. Our
procedure involved three main steps. In a first step we measured the
throughput of the CBP with a wavelength sampling of \SI{1}{nm} between
350 and \SI{1100}{nm} by illuminating a calibrated solar cell. We then
used the calibrated beam to map the response of the StarDICE telescope
at all wavelengths and at several positions on its primary
mirror. Lastly, we interpolated between the sample points to
synthesize the equivalent response of the telescope to illumination
of its full pupill.

A key aspect in the process of calibrating the CBP beam is that the
sensitive area of the photodiode is large enough to cover entirely the
beam. While the measurement is simple in its principle, special
care must be given to several aspects of the setup and analysis to
reach the required level of accuracy. Most notably: (1) the laser beam must be filtered to improve the purity of its wavelength
composition, even with the mitigation measures we adopted, unavoidable
contamination had to be subtracted in the analysis; (2) the setup must minimize scattered light which affects differently the calibration solar cell and the calibrated telescope; (3) achieving sufficient signal to noise ratio in the solar cell implies to fight with high level of $1/f$ noise (pink noise).

To deal with this issue, we selected a solar cell with high
impedance, and grouped the light deposit in short bursts to reduce the
pink noise effects while illuminating the solar cell.
These bursts were separated by dark periods. Using a good synchronization in time
between the bursts and the photocurrent measurements, 
we could estimate the dark current contribution and correct it.
Additionnaly, we increased the collected flux in the solar cell 
by selecting a larger pinhole, inducing a small but measurable change in the
chromatic throughput of the CBP, which must be accounted for.
We did not found significant non-linearities in our measurement
chain. Repeating the measurements allowed to identify slight
evolution in the throughput which were easily corrected

Most of the measurement time is spent in mapping the telescope
response which scales linearly with the number of filters,
resolution in wavelength and number of mirror samples taken. Our main
4-positions analysis which yields subpercent precision on the
synthesized passbands involved about \num{30000} images. The complete
dataset, including runs for the characterization of systematics,
pinhole intercalibration and focal plane measurements, close \num{200000}
images. With our fully automated setup, collecting this dataset required
7 days of uninterrupted operation. 

Our analysis allowed to determine filter fronts with a precision of
about \SI{0.2}{\nano\meter}, to accurately measure out-of-band leaks at a relative
level of \num{e-3} and revealed an unexpected and strong dependency of
the mirror reflectivity with the incidence angle in the UV. As a
result we were able to show through simulations that the uncertainties in our determination of the
telescope passbands will not contribute more than 1\textperthousand\
uncertainty in the measurement of broadband flux with StarDICE after
calibration by observations of the artificial star. 

The results presented in this paper also serve as a proof of concept for the Rubin CBP, specifically designed for measuring the LSST passbands at the Vera C. Rubin Observatory. Although scaling this setup from a 16'' to a \SI{8}{\meter} telescope will be challenging in many regards (more optical surfaces inducing ghosts, larger filters, more sensors), similar numbers are to be expected for the determination of LSST passbands if the necessary amount of calibration time is provided. This calibration will allow an accurate interpretation of the flux ratios between the established photometric standards and the SNe Ia observed to map the expansion history of the Universe. On the other hand, the development of a traveling version of the CBP is currently undergoing \citep{2024RASTI...3..125S}, in order to calibrate the fitler transmissions of on-going and future SNe Ia surveys.


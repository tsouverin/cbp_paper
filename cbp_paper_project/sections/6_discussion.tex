\section{Conclusion}
\label{sec:discussion}

We determined the response of the 16'' StarDICE telescope using a
collimated beam projector (CBP) powered by a tunable laser. Our
procedure involved three main steps. In a first step we measured the
throughput of the CBP with a wavelength sampling of \SI{1}{nm} between
350 and \SI{1100}{nm} by illuminating a calibrated photodiode. We then
used the calibrated beam to map the response of the photometric
instrument at all wavelengths and at several positions on its primary
mirror. Lastly, we interpolated between the sample points to
synthesize the equivalent response of the instrument to illumination
of its full pupill.

A key aspect in the process of calibrating the CBP beam is that the
sensitive area of the photodiode is large enough to cover entirely the
beam. While the measurement is then simple in its principle, special
care must be given to several aspects of the setup and analysis to
reach the required level of accuracy. Most notably, (1) the laser beam
must be filtered to improve the purity of its wavelength
composition. Even with the mitigation measures we adopted, unavoidable
contamination had to be subtracted in the analysis; (2) the setup must
minimize scattered light which affects differently the calibration
photodiode and the calibrated telescope; (3) achieving sufficient
signal to noise ratio in the calibration photodiode necessitates to
fight hight level of pink noise. This was achieved by selecting higher
impedance cells, increasing the amount of flux in the photodiode by
selecting a larger pinhole, grouping the light deposit in short bursts
separated by dark period and using a good synchronization in time
between the bursts and the photocurrent measurement. Surprisingly, the
change in pinhole also induces a small but measurable change in the
chromatic throughput of the CBP which must be accounted for. We did
not found significant non-linearities in our measurement
chain. Repeating the measurements allowed to identified slight
evolution in the throughput which were easily corrected.

Most of the measurement time is spent in mapping the telescope
response which scales linearly with the number of filters,
resolution in wavelength and number of mirror samples taken. Our main
4-positions survey which yields subpercent precision on the
synthesized passbands involved about 30000 images. The complete
dataset, including runs for the characterization of systematics,
pinhole intercalibration and focal plane survey, is closer to 200000
images. With our fully automated setup, collecting this dataset required
7 days of uninterrupted operation. 

The survey allowed to determine filter fronts with a precision of
about 0.2~nm, to accurately measure out-of-band leaks at a relative
level of \num{e-3} and revealed an unexpected and strong dependency of
the mirror reflectivity with the incidence angle in the UV. As a
result we expect that the uncertainties in our determination of the
telescope passbands will not contribute more than 1\textperthousand\
uncertainty in the measurement of broadband flux with StarDICE after
calibration by observations of the artificial star. Providing the
necessary amount of calibration time, similar numbers are to be
expected for the determination of LSST passbands by the Rubin CBP
allowing accurate interpretation of the ratios of fluxes between the
established photometric standards and the supernovae observed to map
the expansion history of the universe.


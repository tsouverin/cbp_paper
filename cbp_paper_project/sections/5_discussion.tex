\section{Synthesis of the equivalent transmission for full pupil illumination}
\label{sec:discussion}

Both the telescope reflectivity and filter transmission display a very clear radial dependency. While the latter is expected due the interferometric nature of the filters, the origin of the former is not yet understood. We can nevertheless build an empirical model matching the measurement, and compute from there a theoretical "full pupill" transmission by averaging the model over the illuminated portion of the primary mirror. As is demonstrated in Table TX, these two steps are necessary to reach sub-percent accuracy in colors and sub-nm accuracy in central wavelengthes. Last we proceed with propagating the statistical and systematic errors on the final transmission curves. The impact of these errors in the interpretation of broadband photometry is better understood when integrated as errors on the filter normalisation and central wavelength, which can readily be translated in errors in colors and color-terms.


\subsection{Radial model of the instrument transmission}\label{sec:model}

\subsection{Full pupil synthetic transmission curves}

\subsection{Final error budget}


\todo{Direct transmission measurement, ghost brightness model, angle dependence,  Pupil synthesis}
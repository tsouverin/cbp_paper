\section{Introduction}

The calibration of optical wide-field surveys need to reach new levels of precision to meet the requirements of type Ia supernovae cosmology. Type Ia supernovae (SNe Ia) are standard candles, a class of objects with predictable luminosity that are used as probes to characterize dark energy. By measuring the luminosity distance of SNe Ia at different redshifts, we can infer dark energy properties with a Hubble diagram. This distance is obtained by measuring the maximum amplitude of the SN Ia light curve, which is observed within different restframe telescope bands depending on its redshift. Errors in the relative flux calibration between the different bands have a knock-on effect on systematic errors in the Hubble diagram, and then dark energy parameters constraints.

Current analyses are joining observations of several surveys to reduce the statistical uncertainty on the measurement on cosmological parameters (\cite{Brout_2022}; \cite{rubin2023union}). Within the new decade, the number of type Ia supernova observations will grow significantly, especially thanks to the Legacy Survey of Space and Time undertaken by the Vera Rubin Observatory  \citep{lsst}, and the statiscial uncertainty on cosmoligical parameters will consequently diminishing. In recent studies (\cite{Betoule_2014}; \cite{Scolnic_2018}), it has been shown that the contribution of flux calibration uncertainty is at a comparable level to the statistical uncertainty.  To benefit from this incoming statistics, the photometric calibration of these instruments needs to reach sub-percent accuracy.

SN Ia survey bandpasses are calibrated relatively to the CALSPEC library of spectrophotometric standard stars (see~\cite{Bohlin_2020}), which rely on radiative transmission model of white dwarf atmospheres (\cite{Narayan_2019}). To transfer this calibration to the instrument survey, it is necessary to take in account two components: (i) the terrestrial atmospheric transmission, and (ii) the central wavelength position of the survey bandpasses. The former can be infered by airmass regression with observations of CALSPEC references stars, while the latter need beforehand precise measurements of the bandpasses throughput. Multiple strategies have been developed to provide this measurement on different surveys. The most common approach consists on using calibrated sensor with extreme precision such as the ones supplied by the NIST \citep{houston2008detectors} to monitor a light source used to illuminate a telescope throughput and its filter transmissions. Several approaches are involving diffusion on a flat-field screen toward the instrument (\cite{stubbs2006}; \cite{marshall2013}). Other designs have been developed, like \cite{Lombardo_2017} which is involving integrating spheres and parabollic mirrors to redirect the light in a parallel beam.

The StarDICE experiment (see~\cite{Betoule_2023}) proposes a metrology chain from laboratory flux references toward a stable light source positioned far enough (\textasciitilde {\SI{100}{\meter}) to appear as a pointlike source for the small astronomical telescope observing (\textasciitilde {\SI{40}{\centi\meter} diameter) and provides in-situ calibration of the instrument. The calibrated light source will emits broadband light that will monitor the $ugrizy$ filters of the StarDICE telescope at the millimagnitude level. Beforehand, it is prefered to have a laboratory measurement of the filters transmission at high wavelength resolution to cross calibrate the measurements. In this context, a Collimated Beam Projector (hereafter CBP) has been developed to accurately measure the StarDICE telescope response $\Rtel$, including the optics, the CCD camera quantum efficiency and the filter transmissions. 

The CBP has originally been designed for the LSST telescope (see \cite{ingraham2016}), and some prototypes have been used to measure the throughput of the CTIO \SI{0.9}{\meter} telescope and the DECam wide field imager, respectively in \cite{coughlin2018} and \cite{coughlin2016}. Compared to flat-field illumination devices, the CBP is a device able to project monochromatic in a collimated beam, which has the main advantage to avoid light scattering into the telescope we want to measure. It is also a necessary feature to identify ghosts in the focal plane due to internal reflections of the optic, which can be a high source of uncertainty if neglected. Yet, such a device is illuminating a fraction of the entrance pupil, so it is needed to scan different positions to synthetize the full-pupil illumination throughput. A first prototype of the CBP for StarDICE has been developed in \cite{Mondrik_2023} as a proof-of-concept, and measured the instrument throughput with a precision of \textasciitilde 3\% for wavelengths between \SI{400}{\nano\meter} nand \SI{800}{\nano\meter}, and a wavelength calibration estimated at \textasciitilde 0.1\%. 

The present paper details the enhanced version of this CBP using a tunable laser as a monochromatic source, injecting light into an integrating sphere with an electrometer and a spectrograph as monitoring instrument. PLaced at the focal point of a Ritchey-Chrétien telescope mounted background, the output light is collimated and illuminate the StarDICE telescope pupil. Combining the calibration effort provided by \cite{houston2008detectors} and \cite{solarcell} for the detectors, and all the lessons learnt from previous prototypes, we aim to measure the StarDICE telescope throughput and "ugrizy" filter transmissions at the sub-percent level, and{ to confirm the wavelength calibration accuracy of 0.1\%. This study has two major intents: (i) provide a first measurement of the StarDICE filter transmission to contribute to the metrology chain of the exepriement, and (ii) plays the role of pathfinder for the future measurement of the LSST telescope with its own CBP version.

The following sections of this paper of structured as follows. Section~\ref{sec:setup} is presenting the laboratory setup of the experiment and give an overview of the campaing of measurements. Section~\ref{sec:rcbp} details the measurements of the CBP optics response. Section~\ref{sec:rsd} presents the obtain measurements of StarDICE telescope throughput and filter transmissions. The StarDICE full-pupil synthesising methodology, and main results are presented in Section~\ref{sec:pupil_stitching}. These results are finally discussed in Section~\ref{sec:discussion}.


 


% The knowledge of the relative transmission of the filters is thus necessary to account for the redshift effect on SNe Ia colors and therefore constrain the dark energy parameters.
%The Collimated Beam Projector (CBP) is a device that measures the transmission of a telescope and its filters. It is composed of a tunable monochromatic laser (Ekspla NT252), that emits light within a range of [350 - 1100] nm, with a resolution of 1 nm and an accuracy of 1 Å. The
%light is injected with an optical fiber into an integrating sphere, whose output is a pinhole of
%variable size, producing a monochromatic, homogeneous and pointlike lightsource. The pinhole
%is set at the focal point of a 152mm Ritchey-Chrétien a mounted backwards, thus providing a
%parallel beam. A photodiode and a spectrograph monitor the surface brightness and wavelength
%of the light inside the sphere. The response of the CBP optical device R CBP (λ) is measured
%by shooting directly into a flux calibrated solar cell 2 , as the ratio of the charges detected in the
%solar cell over the charges detected the monitoring photodiode (fig.2).
\section{Introduction}

The calibration of optical wide-field surveys needs to reach new levels of precision to meet the requirements of type Ia supernovae cosmology. Type Ia supernovae (SNe Ia) are standard candles, a class of objects with predictable luminosity that are used as probes to characterize dark energy. By measuring the luminosity distance of SNe Ia at different redshifts, we can infer dark energy properties. This distance is obtained by measuring the maximum amplitude of the SN Ia light curve, which is observed within different observer frame telescope bands depending on its redshift. Errors in the relative flux calibration between the different bands have a knock-on effect on systematic errors in the Hubble diagram, and then dark energy parameters constraints.

Current analyses are joining observations of several surveys to reduce the statistical uncertainty on the measurement of cosmological parameters (\citealt{Betoule_2014,Scolnic_2018,Brout_2022,rubin2023union}). Within the next decade, the number of type Ia supernova observations will grow significantly, especially thanks to the Legacy Survey of Space and Time undertaken by the Vera Rubin Observatory  \citep{lsst}, and the statistical uncertainty on cosmological parameters will consequently diminish. To benefit from these incoming statistics, the photometric calibration of future surveys needs to reach sub-percent accuracy.

SN Ia survey bandpasses are calibrated relatively to the CALSPEC catalog of spectrophotometric standard stars (see~\cite{Bohlin_2020}), which relies on the radiative transmission model of white dwarf atmospheres (\cite{Narayan_2019}). To transfer this calibration to the photometric survey, it is necessary to take into account two additional components: (i) the terrestrial atmospheric transmission, and (ii) the central wavelength position of the survey bandpasses. The former can be inferred by airmass regression with observations of CALSPEC reference stars, or by slitless spectrophotometric analysis with a dedicated telescope such as the AuxTel, as detailed in \cite{Neveu_2024}. The latter needs beforehand precise measurements of the bandpasses throughput. Multiple strategies have been developed to provide this measurement on different surveys. The most common approach consists of using calibrated sensors such as the ones supplied by the NIST \citep{houston2008detectors} to monitor a light source used to illuminate a telescope throughput and its filter transmissions. Several approaches are involving diffusion on a flat-field screen toward the instrument \citep{stubbs2006,marshall2013}. Other designs have been developed, like \cite{Lombardo_2017} which involves integrating spheres and parabolic mirrors to redirect the light in a parallel beam.

The StarDICE experiment (see~\citealt{Betoule_2023}) proposes a metrology chain from laboratory flux references toward the measurement of standard star spectra. Several steps are needed, to calibrate more and more sensitive detectors, and finally port this calibration to on-sky sources. The StarDICE \SI{40}{\centi\meter} diameter telescope is calibrated with a stable light source positioned far enough (\textasciitilde {\SI{100}{\meter}) to appear as a pointlike source, and provides in-situ calibration of the instrument. The calibrated light source will emit LED light that will monitor the $ugrizy$ filters of the StarDICE telescope at the millimagnitude level. Beforehand, it is necessary to have a laboratory measurement of the filter transmission at high wavelength resolution to interpret the broadband LED and star measurements. In this context, a Collimated Beam Projector (hereafter CBP) has been developed to accurately measure the StarDICE telescope response $\Rtel(\lambda)$, including the optics, the CCD camera quantum efficiency, and the filter transmissions. 

The CBP was originally designed for the LSST telescope (see \citealt{ingraham2016}), and some prototypes have been used to measure the throughput of the CTIO \SI{0.9}{\meter} telescope and the DECam wide field imager, respectively in \cite{coughlin2018} and \cite{coughlin2016}. Compared to flat-field illumination devices, the CBP is a device able to project monochromatic light in a collimated beam, which provides parallel monochromatic illumination over a portion of the primary mirror. A first prototype of the CBP for StarDICE has been developed in \cite{Mondrik_2023} as a proof-of-concept, and measured the instrument throughput with a precision of \textasciitilde 3\% for wavelengths between \SI{400}{\nano\meter} and \SI{800}{\nano\meter}, and a wavelength calibration estimated at \textasciitilde \SI{0.2}{\nano\meter}. 

The present paper details the enhanced version of this CBP using a tunable laser as a monochromatic source, injecting light at the focal point of a Ritchey-Chrétien telescope mounted backward. The output light of the CBP is collimated and illuminates the StarDICE telescope pupil to measure its throughput. Combining the calibration effort provided by \cite{houston2008detectors} and \cite{solarcell} for the detectors, and all the lessons learned from previous prototypes, we aimed at measuring the StarDICE telescope throughput and "ugrizy" filter transmissions at the sub-percent level. The other goal is to confirm the wavelength calibration accuracy of \SI{0.2}{\nano\meter}. This study has two major intents: (i) provides a first measurement of the StarDICE filter transmission to contribute to the metrology chain of the experiment, and (ii) plays the role of pathfinder for the future measurement of the LSST telescope with its own CBP version.

The following sections of this paper of structured as follows. Section~\ref{sec:setup} presents the laboratory setup of the experiment and gives an overview of the campaign of measurements. Section~\ref{sec:rcbp} details the measurements of the CBP optics response. Section~\ref{sec:rsd} presents the obtained measurements of StarDICE telescope throughput and filter transmissions. The StarDICE full-pupil synthesizing methodology and main results are presented in Section~\ref{sec:pupil_stitching}. These results are finally discussed in Section~\ref{sec:discussion}.



% The knowledge of the relative transmission of the filters is thus necessary to account for the redshift effect on SNe Ia colors and therefore constrain the dark energy parameters.
%The Collimated Beam Projector (CBP) is a device that measures the transmission of a telescope and its filters. It is composed of a tunable monochromatic laser (Ekspla NT252), that emits light within a range of [350 - 1100] nm, with a resolution of 1 nm and an accuracy of 1 Å. The
%light is injected with an optical fiber into an integrating sphere, whose output is a pinhole of
%variable size, producing a monochromatic, homogeneous and pointlike lightsource. The pinhole
%is set at the focal point of a 152mm Ritchey-Chrétien a mounted backwards, thus providing a
%parallel beam. A photodiode and a spectrograph monitor the surface brightness and wavelength
%of the light inside the sphere. The response of the CBP optical device R CBP (λ) is measured
%by shooting directly into a flux calibrated solar cell 2 , as the ratio of the charges detected in the
%solar cell over the charges detected the monitoring photodiode (fig.2).
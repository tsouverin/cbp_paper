%& -shell-escape
%\documentclass[a4paper]{aa}
\documentclass[printer]{aa}
% \documentclass{article}
\usepackage[pagebackref,colorlinks,citecolor=blue,urlcolor=blue,linkcolor=blue]{hyperref}
\usepackage[varg]{txfonts}
\usepackage{graphicx}
\usepackage{natbib}
\usepackage{amssymb}
\usepackage{amsmath}
\usepackage{tabularx}
\usepackage{hhline}
\usepackage{makecell}
\usepackage{placeins}
\usepackage{tikz}
\usepackage{subcaption}
\usepackage{lipsum}
\usetikzlibrary{arrows,positioning,shapes} %,optics}
\usepackage{xspace}
\usepackage{mathtools}


%\usepackage[squaren,Gray]{SIunits}
\usepackage{siunitx}
%\usepackage{units} % nicefrac
%\newcommand{\units}[1]{$\mathrm{#1}$}

%instrument 
\newcommand{\SD}{StarDICE\xspace}
\newcommand{\Oldthorlabs}{SM05PD1B\xspace}
\newcommand{\Newthorlabs}{SM05PD3A\xspace}
\newcommand{\Qsolar}{Q_{\mathrm{solar}}}
\newcommand{\Qsolarmes}{Q_{\mathrm{solar}}^{\mathrm{mes}}}
\newcommand{\Qsolarcal}{Q_{\mathrm{solar}}^{\mathrm{cal}}}

\newcommand{\Qphot}{Q_{\mathrm{phot}}}
\newcommand{\Qphotmes}{Q_{\mathrm{phot}}^{\mathrm{mes}}}
\newcommand{\Qphotcal}{Q_{\mathrm{phot}}^{\mathrm{cal}}}
\newcommand{\Qphotcont}{Q_{\mathrm{phot}}^{\mathrm{532}}}

\newcommand{\Qccd}{Q_{\mathrm{ccd}}}
\newcommand{\Qccdcal}{Q_{\mathrm{ccd}}^{\mathrm{cal}}}
\newcommand{\Qccdmes}{Q_{\mathrm{ccd}}^{\mathrm{mes}}}
\newcommand{\Qccdcont}{Q_{\mathrm{ccd}}^{\mathrm{532}}}

\newcommand{\Qspectro}{Q_{\mathrm{spectro}}}
\newcommand{\Qspectrocont}{Q_{\mathrm{spectro}}^{\mathrm{532}}}
\newcommand{\Qspectromain}{Q_{\mathrm{spectro}}^{\mathrm{main}}}

\newcommand{\Ssolarstat}{\sigma_{\mathrm{solar}}^{\mathrm{stat}}}
\newcommand{\Sphotstat}{\sigma_{\mathrm{phot}}^{\mathrm{stat}}}
\newcommand{\Qghost}{Q_{\mathrm{ghost}}}
\newcommand{\Rspectro}{R_{\mathrm{spectro}}}
\newcommand{\Rcbp}{R_{\mathrm{CBP}}}
\newcommand{\Rtel}{R_{\mathrm{tel}}}

\newcommand{\Esolar}{\epsilon_{\mathrm{SC}}}
\newcommand{\Espectro}{\epsilon_{\mathrm{spectro}}}
\newcommand{\Espectrocont}{\epsilon_{\mathrm{spectro}}^{\mathrm{532}}}
\newcommand{\Ephot}{\epsilon_{\mathrm{phot}}}
\newcommand{\Ephotcont}{\epsilon_{\mathrm{phot}}^{\mathrm{532}}}
\newcommand{\Eccd}{\epsilon_{\mathrm{ccd}}(\lambda)}

\newcommand{\spinhole}{\SI{75}{\micro\meter}\xspace}
\newcommand{\bpinhole}{\SI{5}{\milli\meter}\xspace}

\newcommand{\mes}{{}_{\mathrm{, \, mes}}}
\newcommand{\cont}{{}_{\mathrm{, \, 532}}}  %cont plutôt que 532 ?
\newcommand{\contcomp}{{}_{\mathrm{, \, comp}}}  %cont plutôt que 532 ?

\newcommand{\Kghost}{K_{G_1/G_0}(\lambda)}
\newcommand{\Kpinholes}{K_\mathrm{\spinhole/\bpinhole}(\lambda)}

\newcommand{\Rwindow}{R_{\mathrm{win}}(\lambda)\xspace}
\newcommand{\Rccd}{R_{\mathrm{ccd}}(\lambda)\xspace}

\newcommand{\bkg}{\mathrm{bkg}}
\newcommand{\Kghostfit}{K_{G/A}}
\newcommand{\Kghostfitfirst}{K_{G_1/A}}

%\newcommand{\Kg_fit}{\mathrm{K_{G_{(1, 2)}/A}}}

\newcommand{\todo}[1]{\textbf{\textcolor{red}{[#1]}}\xspace}
\newcommand{\com}[1]{\textbf{\textcolor{orange}{(#1)}}\xspace}
\newcommand{\up}[1]{\textsuperscript{#1}}
\renewcommand\tabularxcolumn[1]{m{#1}}


\usepackage[outputdir={fig},debug]{dot2texi}
\usepackage{draftwatermark}
\usepackage{xspace}
\SetWatermarkText{DRAFT}
\SetWatermarkScale{6}
\SetWatermarkLightness{0.9}
\newcommand{\angexp}{\circ{a}}
\newcommand{\texp}{\ensuremath{\tau}\xspace}
\newcommand{\FixMe}[1]{\textbf{\textcolor{red}{[#1]}}\xspace}
\newcommand{\SZF}[1]{\textbf{\textcolor{blue}{[#1]}}\xspace}
\newcommand{\MARC}[1]{\textbf{\textcolor{green}{[#1]}}\xspace}
\newcommand{\THIERRY}[1]{\textbf{\textcolor{green}{[#1]}}\xspace}

\newcolumntype{M}[1]{>{\centering\arraybackslash}m{#1}}
\renewcommand{\arraystretch}{2}
\setlength{\arrayrulewidth}{1.25pt}

\graphicspath{{fig/}}
\makeatletter
\newcommand*\ExpandableInput[1]{\@@input#1 }
\makeatother
\DeclareMathOperator*{\med}{med}
\title{The StarDICE absolute flux calibration experiment: Characterisation of the photometric instrument with a collimated beam projector}

\author{Collaboration Boston Paris}

\abstract{The measurement of magnitudes with different filters in photometric surveys gives access to cosmological distances and parameters. However, for current and future large surveys like the ZTF, DES, HSC or LSST, the photometric calibration uncertainties are becoming greater than statistical uncertainties in the error budget of type Ia cosmology analysis, which limits our ability to use type Ia supernovae for precision cosmology.}{To reach the sub-percent precision over magnitude measurements, it is necessary to have a knowledge of the survey filter bandpasses at the per-mil level. The Collimated Beam Projector (CBP), an optical device able to calibrate the throughput of an instrument and its filter, has been developed to answer this issue.}{We built a CBP with a tunable laser source and a reversed telescope to emit a parallel monochromatic light beam monitored in flux and wavelength. We tested its performance on the measurement of the StarDICE telescope filters as a proof of concept.}{We obtained measurements on StarDICE telescope throughput and its fitler transmissions. After carefully analyzing the systematic uncertainties, we have reach a sub-nanometer accuracy on the filter central wavelengths and detected out-of-band leakages at the sub-per mil level. In the end, we synthesized the equivalent transmission for full pupil illumation of the StarDICE telescope.}{The results have achieve the objectives identified to reduce photometric calibration, and have demonstrate the feasibility of measuring filter bandpasses position of a telescope at the per-mil level. New versions of CBP are developed to improve the setup, notably by changing the lightsource to build a portable version as detailed in \citep{sommer2023design}.}

\begin{document}

\maketitle

%\section{Currently happening}
%
%\begin{itemize}
%\item SC linearity
%\item pinhole choice
%\item dark current caracterization
%\item Photo-diode change
%\end{itemize}

\tableofcontents

\section{Introduction}

The calibration of optical wide-field surveys needs to reach new levels of precision to meet the requirements of type Ia supernovae (SNe Ia) cosmology. SNe Ia are standard candles, a class of objects with predictable luminosity used as probes to characterize dark energy in the late Universe. We can infer dark energy properties by measuring the luminosity distance of SNe Ia at different redshifts. This luminosity distance is obtained by measuring the maximum amplitude of the SN Ia light curve, which is observed within different optical bands depending on its redshift. Errors in the relative flux calibration between the different bands have a knock-on effect on systematic errors in the Hubble diagram, which are then propagated to dark energy parameters constraints.

Current photometric surveys like the Dark Energy Survey (DES) \citep{Brout_2019} or Subaru Hyper Suprime-Cam (HSC) \citep{hsc_2019} observed hundreds of SNe Ia. The Zwicky Transient Facility (ZTF) \citep{ztf_2022} should total up to about \num{10000} spectroscopically confirmed SNe Ia, a consequent increase compared to previous surveys. Joint analysis has been performed to benefit from the different existing surveys, reducing the statistical uncertainty on the measurement of cosmological parameters (\citealt{Betoule_2014,Scolnic_2018,Brout_2022,rubin2023union}). Additionally, we expect the SNe Ia catalog to reach a new order of magnitude within the next decade thanks to the Legacy Survey of Space and Time (LSST) undertaken by the Vera Rubin Observatory \citep{lsst}, which is expected to observe between 120,000 and 170,000 SNe Ia up to redshifts $z \sim 0.3$ \citep{lsst_2022}. With this tremendous increase, the statistical uncertainty on cosmological parameters will consequently diminish, leaving the photometric calibration as one primary source of systematic uncertainty. Therefore, the photometric calibration needs to reach sub-percent precision to benefit from the incoming statistics of the present and future surveys.

SNe Ia survey bandpasses are calibrated relatively to the CALSPEC catalog of spectrophotometric standard stars (\cite{Bohlin_2020}), which relies on the radiative transmission model of white dwarf atmospheres (\cite{Narayan_2019}). To transfer this calibration to the photometric survey, it is necessary to consider two additional components: (i) the terrestrial atmospheric transmission and (ii) the survey filter transmissions as a function of wavelength, with particular attention given to the bandpasses edges wavelength position. The former can be inferred by airmass regression with observations of CALSPEC reference stars or by slitless spectrophotometric analysis with a dedicated telescope such as the Rubin Observatory's auxiliary telescope (AuxTel), as detailed in \cite{Neveu_2024}. The latter needs beforehand precise measurements of the bandpasses throughput. Multiple strategies have been developed to provide this measurement on different surveys. The most common approach consists of using calibrated sensors such as the ones supplied by the National Institute of Standards and Technologies (NIST) \citep{houston2008detectors} to monitor a light source used to illuminate a telescope to measure its throughput and filter transmissions. Several approaches involve diffusion on a flat-field screen toward the instrument \citep{stubbs2006,marshall2013}. Other designs have been developed, like \cite{Lombardo_2017}, which involves integrating spheres and parabolic mirrors to redirect the light in a parallel beam.

The StarDICE experiment (\citealt{Betoule_2023}) proposes a metrology chain from laboratory flux references toward the measurement of standard star spectra. Several steps are needed to calibrate increasingly sensitive detectors and finally transfer this calibration to on-sky sources. The StarDICE \SI{40}{\centi\meter} diameter telescope is calibrated with a stable light source positioned far enough (\textasciitilde {\SI{100}{\meter}) to appear as a pointlike source and provides in-situ calibration of the instrument. Composed of LEDs, the calibrated light will emit broadband flux, which will be used to monitor the $ugrizy$ filters of the \SD{} telescope at the millimagnitude level. Beforehand, it is necessary to have a laboratory measurement of the filter transmission at high wavelength resolution to interpret the broadband LED and star measurements. In this context, a Collimated Beam Projector (hereafter CBP) has been developed to accurately measure the StarDICE telescope response $\Rtel(\lambda)$, including the optics, the CCD camera quantum efficiency, and the filter transmission. 

The CBP was initially designed for the LSST telescope (\citealt{ingraham2016}), and prototypes have been used to measure the throughput of the CTIO \SI{0.9}{\meter} telescope and the DECam wide field imager, respectively, in \cite{coughlin2018} and \cite{coughlin2016}. Compared to flat-field illumination devices, the CBP can project monochromatic light in a collimated beam, which provides parallel monochromatic illumination over a portion of the primary mirror. A first prototype of the CBP for StarDICE has been developed in \cite{Mondrik_2023} as a proof-of-concept and measured the instrument throughput with a precision of \textasciitilde 3\% for wavelengths between \SI{400}{\nano\meter} and \SI{800}{\nano\meter}, and a wavelength calibration estimated at \textasciitilde \SI{0.2}{\nano\meter}. 

This paper details the enhanced version of the StarDICE CBP, now equipped with a tunable laser as a monochromatic source that injects light at the focal point of a Ritchey-Chrétien telescope mounted backward. The collimated CBP output beam illuminates the StarDICE telescope pupil to measure its throughput. Incorporating lessons from previous CBP prototypes, and leveraging calibration efforts provided by \cite{houston2008detectors} and \cite{solarcell} for the detectors, our goal is to achieve sub-percent precision in measuring the StarDICE telescope throughput and $ugrizy$ filter transmissions while confirming the \SI{0.2}{\nano\meter} wavelength calibration accuracy. This study has two major intents: (i) providing a first measurement of the StarDICE filter transmission to contribute to the StarDICE experiment's metrology chain, and (ii) serving as a pathfinder for the future measurement of the LSST telescope with its dedicated version of the CBP.

The study follows three main steps. First, we illuminate a calibrated solar cell with the CBP to determine its throughput, using a wavelength sampling of \SI{1}{nm} across the 350 to \SI{1100}{nm} range. Second, we direct the calibrated CBP beam toward the StarDICE telescope to measure its throughput across the same wavelength range. Since the CBP output does not fully illuminate the StarDICE primary mirror, we perform a series of measurements at different positions, repeating this step for each position. Finally, we reconstruct the telescope's full-pupil response by interpolating the measurements across the scanned positions. Several factors must be carefully controlled to ensure accurate CBP throughput calibration, including laser spectral purity, minimizing scattered light (which differently affects the solar cell and the telescope), and achieving a sufficient signal-to-noise ratio when measuring the CBP output beam, given the high $1/f$ noise (pink noise) in these detectors.

The remainder of this paper details the acquisition plan, hardware setup, and software tools used to achieve a sub-percent calibration of the StarDICE telescope response and filter transmissions. The structure is as follows: Section~\ref{sec:setup} describes the laboratory setup and measurement campaign. Section~\ref{sec:rcbp} details CBP optical response measurements. Section~\ref{sec:rsd} presents the measured StarDICE telescope throughput and filter transmissions. Section~\ref{sec:pupil_stitching} discusses the methodology for synthesizing the full-pupil response, with results and implications discussed in Section~\ref{sec:discussion}.

\section{Laboratory setup}
\label{sec:setup}

Our setup is composed of three main components: the \SD telescope, a large area photodiode used as a calibration reference, and the CBP which illuminates either one or the other. In the following sections, we will describe each component in detail and explain their operation.

\subsection{\SD}
\label{sec:stardice}

The \SD photometric instrument consists in a Newton
telescope with a primary mirror of \SI{40}{\centi\meter} diameter
(16'') and \SI{1.6}{\meter} focal length ($f/D = 4$). The focal plane
hosts an Andor Ikon-M DU934P-BEX2-DD camera, equipped with a deep
depleted and back-illuminated CCD sensor (E2V DU934P). The active area
of the sensors is
$\SI{13.3}{\milli\meter}\times\SI{13.3}{\milli\meter}$ divided in
$1024\times 1024$ square pixel of \SI{13}{\micro\meter} side. In this
baseline setup, the pixel resolution is \SI{1.68}{\arcsec} and the
field of view $\SI{28.6}{\arcmin}\times\SI{28.6}{\arcmin}$.

A 9-slots \SI{28.5}{\milli\meter} filter-wheel in front of the camera
features 6 interference filters in the $ugrizy$ photometric system, a
Star Analyser 200 diffraction grating, and a \SI{0.2}{\milli\meter}
pinhole. The remaining slot is left empty. Aside the optional filters,
the only other glass part in the light-path is the non-coated
fused-silica window of the CCD cell\footnote{The manufacturer code for
  this window is WN35FS(BB-VV-NR)W}. The two sides of the window are
affected by a \SI{0.5}{\degree} wedge.

The z-position of camera-filterwheel assembly is adjustable over
\SI{9}{\centi\meter} allowing to focus from distances as close as
\SI{35}{m} up to infinity. The 11cm diagonal flat is oversized to
ensure the fully-illuminated plane extends over the sensor with a
comfortable margin in all optical configurations. A model of the
baffling and optics of the instrument build upon the
batoid\footnote{\url{https://github.com/jmeyers314/batoid}} raytracing
software has been tuned on pinhole images and is described in a
companion technical note \cite{}.

In operation, the camera sensor is thermoelectrically cooled-down to a
temperature of \SI{-70}{\celsius}, delivering a median dark-current of
\SI{0.15}{e^-/s}, neglected for the $\sim 1s$ exposures considered in
this study. When operated in the lab, the telescope was mounted on a
custom altazimutal mount to enable easy alignment with the CBP.

\subsection{Collimated Beam Projector}
\label{sec:cbp}

The Collimated Beam Projector (CBP) general setup needs the following: one tunable monochromatic light source, and an optic device able to recreate a parallel beam from a point source. 
In our case, the light source is an Ekspla NT252 tunable laser, using a Q-switched pump laser at \SI{532}{\nano\meter} and non-linear crystals to produce powerful monochromatic pulses from 335 to \SI{2600}{\nano\meter}. The pulse duration is fixed and lies between 1 to \SI{4}{\nano\second} with an energy of \SI{1.1}{\milli\joule} in the near infra-red. The energy can be decreased at will by a factor of 2 using the tuning of the Q-switch (namely QSW in the following) which degrades the quality factor of the resonant cavity. The pulses are shot with a fixed frequency of \SI{1}{\kilo\hertz} and can be shot in two modes. The first one is said "continuous", so it shoots pulses in a continuous way. The second one send packets of pulses by "bursts". Each burst is composed from 1 to 1000 pulses, meaning the duration of a burst is restrained between \SI{1}{\milli\second} to \SI{1}{\second}. The pulses have a maximum linewidth below \SI{10}{\per\cm}, which can be converted into a maximum spectroscopic width below \SI{0.4}{\nano\meter} around \SI{600}{\nano\meter}. This is an upper limit quoted by the manufacturer, but measurements taken using a similar laser show bandwidths going from \SI{0.08}{\nm} to \SI{0.48}{\nm} in the 350 to \SI{1100}{\nm} range \citep{woodward2018}. The standard deviation of pulse energy is around 2.5\%. This laser is composed of three stages to operate (from 335 to \SI{669}{\nano\meter}, from 670 to \SI{1064}{\nano\meter} and above \SI{1064}{\nano\meter}), which results in three different regimes of power and light contamination in the CBP. This laser was chosen for the offered wide wavelength range and the high pulse energy, as source power is an important criteria for the CBP calibration with the solar cell. 

The laser output is purified with a diaphragm and a filter wheel. The latter contains three different broad bandpass filters that helps purifying the laser light from pump photons or other resonances in the system. In particular, we use a pass-blue filter BrightLine Multiphoton FF01-680/SP-25 in the regime 530 to \SI{645}{\nano\meter} to remove \todo{a bien justifier}, a pass-red filter RazorEdge LP03-532RU-25 to filter the \SI{532}{\nano\meter} pump photons in the regime 645 to \SI{1074}{\nano\meter}, and a pass-infrared filter RazorEdge LP02-1064RU-25 above \SI{1074}{\nano\meter} to filter $<\SI{1064}{\nano\meter}$ photons appearing in this regime. A black metallic box encloses the entire optical stage to minimize light scattering in the room.

The light is then injected into an optical fiber Thorlabs MHP910L02 of with a wide core diameter of \SI{910}{\micro\meter}, which is plugged into an IS200-4 $\Phi$2'' integrating sphere from Thorlabs, which has an internal \SI{50}{\mm} diameter and is composed of 1 input and 4 output ports. The integrating sphere dilutes the laser flux and breaks the light coherence. Two monitoring instruments are plugged on the sphere. First a silicon photodiode Thorlabs SM05PD3A with a high efficiency in the UV is mounted on an output port orthogonal to the laser input port to control the laser power stability. It is mounted behind a pinhole to reduce the photon flux and ensure that it works in its linear regime. It is read by a Keithley 6514 electrometer at a rate of \SI{0.02}{\second}. We also plug an OceanOptics Q65000 fiber spectrometer on another output to monitor the true laser wavelength and the spectral purity. One port is left free to plug a calibration lamp to calibrate the spectrograph when needed. Finally, a slider with pinholes of different diameters is mounted on the last output port. Figure \ref{fig:sphere} shows a schematic of the integrating sphere and the instruments plugged into it.

In this paper, we used three different pinholes, of diameter \SI{75}{\micro\meter}, \SI{2}{\milli\meter} (respectively P75HK and P2000HK from Thorlabs) \SI{5}{\milli\meter} (homemade). The \SI{5}{\mm} pinhole is the largest possible given the \SD field of view and gives the maximum flux. The other pinholes are used for systematic checks and filter edge analysis. The pinhole slider is attached to the ocular of a 154/1370 Ritchey-Chrétien Omegon telescope, mounted on a robotic Celestron NexStar Evolution 6, in order to position the pinhole at the focal point of the optics. An iris diaphragm is positioned \SI{16}{mm} after the pinhole and adjusted to cut light that would otherwise miss the secondary mirror of the telescope. A small amount of light scatters on the iris blades and is best observed in telescope images of the CBP taken with the \SI{2}{mm} pinhole configuration. In these images, it forms a faint ring with a radius of approximately \SI{340}{pixels}, discinct from the main spot. When comparing annular photometry of the ring to that of the main spot, the fraction of scattered light is usually smaller than \num{6e-4} between 400 and \SI{900}{nm}.

\begin{figure}
    \centering
    \includegraphics[width=1\columnwidth]{fig/integrating_sphere_3d.pdf}
    \caption{Schematic of the integrating sphere}
    \label{fig:sphere}
    %Tinkercad
\end{figure}


One important improvement in our CBP design compared with \cite{Mondrik_2023} is the use of a solar cell to calibrate and monitor the CBP optical transmission between the optical sphere and the telescope output. It was chosen to have high quantum efficiency and a large shunt resistance. Hence we have a C60 solar cell of 3$^{\mathrm{rd}}$ generation from Sunpower. This solar cell is set on a two-axis mount, one that allows a movement in the direction of the optical axis of the CBP telescope, and another axis that is vertical, which allows to adjust the height of the solar cell. It is placed at \SI{16}{\cm} approximately from the telescope aperture. This solar cell is connected via a coaxial cable to a Keysight B2987A electrometer, the same that was used for its calibration. The charges are measured at a rate of \SI{0.002}{\second}. In Figure~\ref{fig:cbp_setup} right, we show a picture of the setup when the CBP is aiming at the \SD telescope described in section~\ref{sec:stardice}. While a large CBP output beam allows for a larger scan of the telescope mirror, its calibration needs for it to be fully contained by the calibrating sensor. The largest accurate sensor we managed to setup for this experiment is a square solar cell of \SI{12.5}{\centi\meter} width, which thus sets the maximum size of the CBP beam. We adjust the output of the Richey-Chrétien Omegon to that size by cropping it with a frame shaped as a quarter of a disk corresponding to a quadrant of its secondary mirror spider. 

To synchronise the laser and the two electrometers on the same clock, their output trigger lines are plugged in a digital analyser that defines the timestamps for all our data sets. This was a crucial element in the chain to achieve an accurate calibration of the CBP response when signal-to-noise ratio was too low in the solar cell depending on the laser regime.

\begin{figure*}[ht]
\centering
\includegraphics[width=\textwidth]{fig/cbp_setup_cropped.pdf}
\caption{Pictures of the different CBP setups. Left: CBP setup when shooting in the solar cell. Right: CBP setup when shooting in the \SD telescope.}
\label{fig:cbp_setup}
\end{figure*}

In summary, the design of the CBP has undergone several modifications since its initial description in \cite{Mondrik_2023}. Three notable changes include: (1) replacing the laser light source from NIST with an EKSPLA NT 252, which has a slightly different wavelength power cutoff; (2) adding a filtering system to prevent laser harmonics from being injected into the light beam; and (3) incorporating calibration and monitoring of the CBP transmission as part of the instrumentation and observation procedure. This is now done using a solar cell instead of relying on external measurements in a separate experimental setup.


\subsection{Solar cell description}

\subsubsection{Quantum efficiency measurement}
 
The quantum efficiency (QE) as a function of wavelength for the solar cell was measured relative to a NIST-calibrated photodiode by using a monochromator as a light source and using an electrometer to measure the current of the solar cell and of the photodiode at each wavelength. The monochromator wavelength was calibrated with a spectrometer relative to a mercury calibration source. Details of the setup are discussed in ~\cite{solarcell}. The quantum efficiency of the solar cell used for these measurements is shown in Fig.~\ref{fig:qe}. The uncertainty cited is from the standard error of the current measurements.

\begin{figure}[!h]
\centering
\includegraphics[width=\columnwidth]{solarcell_qe}
\caption{Solar cell quantum efficiency with respect to wavelength (top) and error bar sizes (bottom).}
\label{fig:qe}
\end{figure}

To determine the effect of temperature on QE, the QE of the solar cell was measured as it was heated. At longer wavelengths, the QE increased slightly as the temperature increased (see Fig.~\ref{fig:SC_temp}). This trend is consistent with the temperature dependence of silicon's QE \citep{Green_2008}. At 1050 nm, the QE changes by more than 0.1 percent per degree.
\begin{figure}[!h]
\centering
\includegraphics[width=\columnwidth]{SC_temp_dep.pdf}
\caption{(a) Solar cell quantum efficiency vs temperature for wavelengths ranging from 600 to 1050 nm. (b) Fitted change in quantum efficiency per degree Celsius for wavelengths ranging from 600 to 1050 nm.}
\label{fig:SC_temp}
\end{figure}

Additionally, the effect of the angle of incidence on the solar cell QE was measured. The solar cell was rotated up to 35 degrees off-axis. The results are shown in Fig.~\ref{fig:SC_angle}. There is a stronger dependence on angle of incidence at wavelengths that are less than 600 nm, but the change in QE for wavelengths greater than 400 nm is less than 5e-4 per degree.
\begin{figure}[!h]
\centering
\includegraphics[width=\columnwidth]{SC_angle_dep.pdf}
\caption{(a) Solar cell quantum efficiency vs solar cell tilt relative to normal incidence for different wavelengths. (b) Fitted change in quantum efficiency per degree of solar cell tilt relative to normal incidence vs wavelength.}
\label{fig:SC_angle}
\end{figure}

The solar cell was aligned perpendicalular to the CBP output beam with a precision better than \SI{1}{\degree} and not moved during all the measurement campaign. Meanwhile, room temperature was monitored and has not varied more than \SI{2}{\degreeCelsius}.
%Considering measured variations of quantum efficiency with angle and temperature, we assumed that this systematic is well below the per-mill level and is not considered in the following of the paper. 

%\subsubsection{Current or charge mode}
%
%The charges in the photodiode and the solar cell are read with electrometers in charge mode. This mode is better than the current mode as the integration is performed in a capacitance. This avoids modeling the precise current timelines before integrating them numerically, as they are subtlety different from squarewave functions due to different time responses in the system and contaminated by random dark current fluctuations. 

%\begin{itemize}
%\item Ekspla NT252 tunable laser
%\item Light injection system with filters
%\item Optical sphere ref
%\item Ocean QE65000 spectrograph
%\item Photodiode + Keithley 6514. Fig \ref{fig:thorlabs_response}\\
%  \emph{Old one} (used in summer 2021 up to 14th of October 2021): Thorlabs
%  SM05PD1B with FDS100 spectral response data\\
%  \emph{New one} (replaced on the 14th of October 2021): Thorlabs SM05PD3A with
%  FD11A spectral response data

%\item Pinhole slider
%\item Telescope: Ritchey Chretien Omegon telescope\\
%  Apperture ratio $f/9$\\
%  Apperture $154 mm$ \\  
%\item Telescope mount
%\item Solar Cell on movable mount
%\item Keysight B2987A: in order to be able to chop the light at a higher frequency
%  than what the Keithley can do. And it is a more contemporary instrument.
%\end{itemize}
  
%\begin{figure}[!ht]
%    \begin{center}
%      \includegraphics[width=0.8\columnwidth]{thorlabs_response}
%    \end{center}
%    \caption[]{Response curves of the photodiodes. We see the noticeable
%      improvement in the blue for the new one.}
%    \label{fig:thorlabs_response}
%\end{figure}

\subsubsection{Dark current characterisation}

Due to the low resistance of the solar cell, when plugged into the Keysight electrometer we observed a quite strong current even in dark conditions (about \SI{20}{\nano\ampere}). Therefore, we built a device that uses a precision voltage source with a tunable voltage divider to inject a counter-current that cancels this contribution. The current value was tuned to observe approximately no drift when using the Keysight in charge mode inside the usual acquisition time window ($< \SI{1}{\minute}$). Doing so, we avoided saturating the electrometer when using the solar cell in dark and laser-on conditions. 

%\begin{itemize}
%\item Without offset box: $\sim 20 nA$
%\item With offset box: $~ <0> nA \pm 5 nA$
%\end{itemize}

After cancelling the dark current drift, the analysis of the solar cell dark signal revealed a $1/f$ noise, with power line harmonics contributions at \SI{50}{\hertz},  \SI{100}{\hertz}, and \SI{150}{\hertz} (see Figure~\ref{fig:darkcurrentspectrum}). 
We checked that connecting the Keysight to a resistor of same value as the solar cell shunt resistance gave exactly the same
power spectrum. Then the dark current comes entirely from the Keysight, and it increases with the shunt resistance. To decrease the contribution of the $1/f$ noise in the solar cell time series, we chose the calibrated solar cell with the highest shunt resistance we had ($R_{\rm shunt} = \SI{1.8}{\kilo\ohm}$) and limited the burst durations to at most 200 pulses (\SI{200}{\ms}). Doing so, in the burst time window, the Keysight noise is dominated by the power line harmonics whereas the random $1/f$ noise remains subdominant.




%cbp_paper_plots.py
\begin{figure}[h]
\begin{center}
\includegraphics[width=1\columnwidth]{sc_dark_ps}
\end{center}
\caption[]{Solar cell dark current power spectrum for 20 runs without light in the solar cell (gray curves) and their average (blue). The light gray region encompasses expected noise spectrum contribution for \SI{200}{\ms} laser bursts.}
\label{fig:darkcurrentspectrum}
\end{figure}

\FloatBarrier   

\subsection{Equations}

%From equations, draw the data taking plan
%with calibrated wavelength

The Table~\ref{tab:quantities} sum up the three quantities that are measured with the two setups in Figure~\ref{fig:cbp_setup}.

\begin{table}
    \centering % used for centering table
    \begin{tabular}{M{1cm} M{7cm}} % centered columns (4 columns)
        \hline\hline % inserts double horizontal lines
        Name & Description \\
        \hline
        \bf{$\Qphot(\lambda)$} & The charge per burst collected by the integrating sphere monitoring photodiode, measured by the Keithley 6514 in Coulomb \\

        \bf{$\Qsolar(\lambda)$} & The charge per burst collected by the solar cell, measured by the Keysight 2987A in Coulomb \\
        \bf{$\Qccd(\lambda)$} & The charge collected by the Andor CCD camera of the \SD telescope in ADU. \\
        \hline %inserts single line
    \end{tabular}
    \caption{Needed quantities to compute CBP and \SD telescope response.}
    \label{tab:quantities} % is used to refer this table in the text
\end{table}

First, we need to measure the response of the CBP optics $\Rcbp(\lambda)$ by shooting into the calibrated solar cell as shown in Figure~\ref{fig:cbp_setup} left. It is computed with the Equation~\ref{eq:rcbp}. In this equation, we know the quantum efficiency of the solar cell $\Esolar$ from \cite{solarcell}, and $e$ is the elemental charge of the electron.

\begin{equation}
    \Rcbp(\lambda) = \frac{\Qsolar(\lambda)}{\Qphot(\lambda) \times \Esolar \times e}
    \label{eq:rcbp}
\end{equation} 

Once we have this response, we can shoot inside the \SD telescope as show in Figure~\ref{fig:cbp_setup} right, and obtain $\Rtel(\lambda)$ with the Equation~\ref{eq:rsd}.

\begin{equation}
    \Rtel(\lambda) = \frac{\Qccd(\lambda)}{\Qphot(\lambda) \times \Rcbp(\lambda)}.
    \label{eq:rsd}
\end{equation}

We will focus on $\Rcbp(\lambda)$ and $\Rtel(\lambda)$ respectively in sections \ref{sec:rcbp} and \ref{sec:rsd}.

\subsection{Strategy}
\label{sec:strategy}

With the CBP, we aim to measure $\Rtel(\lambda)$ at the per mil level. We detail step by step the whole strategy established to reach this goal in Table~\ref{tab:schedule}. Then for each line, we give the label of the measurement, the target aimed with the CBP, the pinhole in the CBP slide, the QSW of the laser, the filters used in the \SD camera filterwheel if the light is shot in the telescope, the specificity of the measurement and the number of runs.

The campaign of measurement has started and ended with a calibration of the spectrograph. A calibration Hg-Ar lamp was plugged into the integrating sphere and its light was measured with the spectrograph. This corresponds to lines No.~1 and No.~13 of the Table~\ref{tab:schedule}.

There are two pinhole sizes in the CBP slide to meet one requirement for each configuration of the setup. When shooting into the \SD telescope, we use the \spinhole pinhole so it is seen as a point-like source by the \SD camera; but when shooting into the solar cell, we use the \bpinhole pinhole to have the best signal-to-noise ratio. Since $\Rcbp(\lambda)$ depends on the pinhole diameter, we need to intercalibrate the two responses $R_\mathrm{CBP}^{\mathrm{\SI{5}{\milli\meter}}} (\lambda)$ and $R_\mathrm{CBP}^{\mathrm{\SI{75}{\micro\meter}}} (\lambda)$. This intercalibration can be performed thanks to the measurements in line No.~8 of Table~\ref{tab:schedule}.

A major issue with the CBP is that its output light does not illuminate the entirety of the \SD primary mirror as an astrophysical source would do, but only a portion of it. It is necessary to perform a \textit{pupil stitching} to reconstruct the transmission of the mirror, by combining the measurement of $\Rtel(\lambda)$ when shooting at different positions on the mirror. When doing so, only the point of impact on the mirror is modified, but the point of incidence on the focal plane is the same. The different positions are shown in Figure~\ref{fig:8_mirror_positions}. The pupil stitching corresponds to lines No.~2 and No.~3 of Table~\ref{tab:schedule}. The method used to perform the pupil stitching is detailed in Section~\ref{sec:pupil_stitching}.

To check the uniformity of the \SD focal plane, a measurement of $\Rtel(\lambda)$ at 16 different positions on the focal plane has been performed. The positions are shown in Figure~\ref{fig:ccd_grid}. Only the position on the focal plane is modified while the point of impact on the mirror stays the same. This dataset corresponds to the line No.~12 of Table~\ref{tab:schedule}. 

Finally, some systematic measurements have been carried out. A cap has been placed on the CBP output to measure the background of the room for both setup configurations, corresponding to lines No.~9 and No.~10 of Table~\ref{tab:schedule}. A measurement of the scattered light is possible with the dataset No.~11, by shifting the position of the solar cell from the CBP output of approximately \SI{16}{\centi\meter}


\begin{table*}[t]{}
    \centering
    \begin{tabular}{M{.25cm} M{2.25cm} M{1.75cm} M{1.1cm} M{1.75cm} M{3cm} M{3.5cm} M{1.25cm}}
        \hline\hline
         \bf{N$^{\circ}$} & \bf{Label} & \bf{Target} & \bf{Pinhole} & \bf{QSW} & \bf{\SD bands} & \bf{Specitifity} & \bf{Number of runs} \\ 
         \hline
         1 & Wavelength calibration & Spectrograph & - & - & - & Hg-Ar lamp used for light source & 1 \\ 
         
         2 & Radial pupil stitching & \SD & \SI{75}{\micro\meter} & MAX & \shortstack{u, g, r, i, z, y, \\ EMPTY, GRATING} & 4 radial positions on the mirror  & 1 per position \\
         
         3 & Quadrant pupil stitching & \SD & \SI{75}{\micro\meter} & MAX & EMPTY & 4 quadrant positions on the mirror  & 1 per position \\
         
         4 & Repeatability measurement & \SD & \SI{75}{\micro\meter} & MAX & \shortstack{u, g, r, i, z, y, \\ EMPTY, GRATING} & Same position on the mirror and focal plane for \SI{75}{\micro\meter} pinhole & 3 \\
         
         5 & CBP response calibration before & Solar cell & \SI{5}{\milli\meter} & 298, MAX & - & Monitor the CBP response before the \SD measurement & 5 \\
         
         6 & \SD main calibration & \SD & \SI{5}{\milli\meter} & MAX & \shortstack{u, g, r, i, z, y, \\ EMPTY, GRATING} & Same position on the mirror and focal plane for \SI{5}{\milli\meter} pinhole & 5 \\
                  
         7 & CBP response calibration after & Solar cell & \SI{5}{\milli\meter} & 298, MAX & - & Monitor the CBP response after the \SD measurement & 5 \\
         
         8 & Pinholes inter-calibration & \SD & \SI{75}{\micro\meter}, \SI{2}{\milli\meter}, \SI{5}{\milli\meter} & MAX & EMPTY & 3 different pinhole sizes & 1 per pinhole \\
            
         9 & \SD background measurements & \SD & \SI{5}{\milli\meter} & MAX & EMPTY & Cap on CBP output to block the light & 1 \\
            
         10 & Solar cell background measurements & Solar cell & \SI{5}{\milli\meter} & MAX & - & Cap on CBP output to block the light & 2 \\
            
         11 & Solar cell distance calibration & Solar cell & \SI{5}{\milli\meter} & 298, MAX & - & 2 solar cell positions at \SI{16}{\centi\meter} relative distance & 1 per position \\
            
         12 & Focal plane measurement & \SD & \SI{75}{\micro\meter} & MAX & EMPTY & 4x4 grid positions on the \SD focal plane & 1 per position \\
           
         13 & Wavelength calibration & Spectrograph & - & - & - & Hg-Ar lamp used for light source & 1 \\ 
         \hline
    \end{tabular}
    \caption{Detailed schedule of the measurements.}
    \label{tab:schedule}
\end{table*}


\begin{figure}[!h]
\centering
\includegraphics[width=\columnwidth]{fig/8_mirror_positions.pdf}
\caption{Left: Schematic of the 4 different radial relative positions on the primary mirror of the \SD telescope. Right: Schematic of the 4 different quadrant relative positions on the primary mirror of the \SD telescope.}
\label{fig:8_mirror_positions}
\end{figure}

\begin{figure}
    \centering
    \includegraphics[width=\columnwidth]{fig/ccd_grid_colors.pdf}
    \caption{Schematic of the (4x4) grid positions on the \SD CCD focal plane.}
    \label{fig:ccd_grid}
\end{figure}

\newpage

\section{CBP response calibration with a solar cell}
\label{sec:rcbp}

\subsection{Preparatory settings}

Before illuminating the solar cell with the CBP, several actions have to be taken to maximise the quality of the data sets:
\begin{itemize}
\item place the pinholes at the focus of the CBP telescope
\item tune the telescope iris
\item align the CBP and the solar cell in position and angle
\item mask every ambient light
\item spectrograph calibration
\item \todo{think a lot about what we forgot}
\end{itemize}


\subsubsection{Iris}

\todo{Marc}

\subsubsection{Focussing}

\todo{Marc}

The CBP was aligned with the StarDICE telescope to improve the focus. The
StarDICE camera was set close to infinity with no filter (focus encoder:
8mm). The smallest pinhole (\SI{75}{\um}). We adjusted the CBP focus to get the smallest extension of the pinhole image. A slight offset was then added to account for the small change in focus introduced by tightening the locking screws. Here is a confirming image taken just after.

The focus was to check to resist slot changes and even complete removal and
reinstallation of the pinhole.

For the record, mount coordinates to get CBP and telescope aligned were
$[13.199987411499023, 1.4999985694885254]$

\subsubsection{Alignment of the CBP}

To measure the CBP response, we must ensure that we aim toward the solar cell. For that, we studied the signal collected in the solar cell $\Qsolarmes$ with respect to the CBP mount coordinates in azimuth and altitude. We defined the position at which we park the CBP mount when no measurement is taken at $(\mathrm{alt} = \SI{0}{\degree}, \mathrm{az} = \SI{0}{\degree})$, and the coordinates in the figure \ref{fig:cross_sc} are relative to this origin. With this figure, we can estimate the coordinates at which $\Qsolarmes$ is maximum, corresponding to the solar cell coordinates in the CBP mount frame of reference. We set the solar cell coordinates at $(\mathrm{alt} = \SI{6}{\degree}, \mathrm{az} = \SI{10}{\degree})$. These coordinates correspond to the point aimed by the CBP optics for any further solar cell analysis in this paper.

\begin{figure}[h]
    \centering
    \includegraphics[width=\columnwidth]{fig/cross_solarcell.pdf}
    \caption{Evolution of the solar cell charge $\Qsolarmes$ with respect to the CBP mount coordinates. Left: evolution with respect to the altitude coordinate. Right: evolution with respect to the azimuth coordinate.}
    \label{fig:cross_sc}
    %/stardice/analysis/cbp_paper/golden_sample_analysis/dr2/cross_solarcell.ipynb
\end{figure}

\subsubsection{Ambient light}

Depending on their direction, ambient light can contaminate sensitive devices in the lab room. We masked all diodes from computers or electronic devices.

\subsubsection{Spectrograph error model estimation and wavelength calibration}

The spectrograph was characterised by taking a series of dark exposures with four different exposure times (the same used in the CBP response measurement) to evaluate its gain and readout noise. The goal is to build an error model for the spectrograph.

For each exposure time, a master dark is constructed, averaging all the spectra. Then, for each spectrograph pixel, we fitted a line through the master dark values as a function of the exposure times. The intercept gave the sensor bias value for each pixel, and a master bias $B(\lambda_p)$ is assembled from the intercept values, with $\lambda_p$ the raw spectrograph wavelength value before any calibration associated with each sensor pixel $p$. 

The master bias was subtracted from all dark exposures. From those data, we measured the readout noise and sensor gain. Let's call $D(\lambda_p)$ the dark exposure minus the master bias value for pixel $p$. For each exposure time, the variance $\sigma_p^2$ and average $\bar{D}(\lambda_p)$ of each $D(\lambda_p)$ pixel were computed. The variance evolution with the average is well described by a second-order polynomial function, parameterised as follows:
\begin{equation}\label{eq:spectro_error_model}
\sigma^2_p =\sigma_{ro}^2 +  \bar{D}(\lambda_p)/G + \sigma_G^2 \bar{D}^2(\lambda_p)
\end{equation}
with $\sigma_{ro}$ the readout noise, $G$ the sensor gain and $\sigma_G$ a statistical noise on the gain itself. The fit of this model to data led to the following values (Figure~\ref{fig:spectro_ptc}):
\begin{align}
    \sigma_{ro} &= 1.26\ \mathrm{ADU}, \\
    G & = 25.8\ e^-/\mathrm{ADU} ,\\
    \sigma_G & = 0.7\%.
\end{align}
The first two values are compatible with the CCD specifications given by the spectrograph vendor. 

%spectrograph_dark_and_bias.py
\begin{figure}[!h]
\centering
\includegraphics[width=\columnwidth]{spectro_ptc}
\caption{Spectrograph photon transfer curve to estimate gain and read-out noise. Pixel variance from dark spectra is represented versus their averages for four different exposure times $\tau_{\mathrm{exp}}$.}\label{fig:spectro_ptc}
\end{figure}


We calibrated the spectrograph according to the manufacturer's specifications before and after the main data acquisition run to measure the CBP and \SD responses. Light from a Hg-Ar lamp was injected into the integrating sphere to illuminate the spectrograph sensor. The master bias $B(\lambda_p)$ was subtracted from all spectra. The latter were stacked, and the spectrograph error model from equation~\ref{eq:spectro_error_model} was used to get a first estimate of the intensity uncertainties. The emission lines were fitted on the stacked spectrum using Gaussian profiles plus a polynomial background. Let's call $\lambda_g$ the line centroid fitted by this Gaussian profile. The fit is unweighted to avoid any dependence of $\lambda_g$ with the line flux and checked it was the case, but a statistical uncertainty $\sigma_\lambda^{\rm noise}$ is still estimated for $\lambda_g$ as 
\begin{equation}\label{eq:sigma_lambda_stat}
    \sigma_\lambda^{\rm noise}= \dfrac{1}{\sum_k F_k} \sqrt{\sum_p \left(\lambda_p - \lambda_g\right)^2 \sigma_p^2}
\end{equation}
with $F_k$ the flux in pixel $k$ after master bias subtraction. The sums are performed over a window of size $\pm 3 \sigma_g$ around $\lambda_g$ where $\sigma_g$ is the fitted line Gaussian profile RMS. Equation~\ref{eq:sigma_lambda_stat} corresponds to the propagation uncertainty formula for the average wavelength weighted by flux $F_p$ in a $\pm 3 \sigma_g$  window around $\lambda_g$. Doing so, we considered the shot-noise without biasing our $\lambda_g$ fit.

To transform raw sensor wavelengths $\lambda_p$ into calibrated wavelengths $\lambda_c$, we fitted a third-order polynomial function as requested by the manufacturer to minimise the distance to Hg-Ar tabulated values $\lambda_t$. In doing so, we minimised the following function 
\begin{equation}
    \chi_\lambda^2(a_3, a_2, a_1, a_0) = \sum_{\text{lines}} \left(\lambda_t-a_3 \lambda_g^3 - a_2 \lambda_g^2-a_1 \lambda_g -a_0\right)^2
\end{equation}
over the four polynomial coefficients $a_3, a_2, a_1$ and $a_0$. The sum is performed over lines with high significance (signal-to-noise ratio above 20), and known doublet lines were excluded. The minimisation leads to the four best-fit parameters $\hat a_i$ associated with their covariance matrix. 
As the initial Gaussian fit was unweighted, the covariance matrix is then re-scaled with a global factor to get a final reduced $\chi_\lambda^2$ of one: let's call it $\mathbf{C}_\lambda$. 
Finally, detected line centroids $\lambda_g$ are transformed into calibrated wavelengths $\lambda_c$ using the third order polynomial function $c(\lambda_g)$ with the four best fit parameters:  
\begin{equation}
    \lambda_c \equiv c(\lambda_g) \equiv \hat a_3 \lambda_g^3 + \hat a_2 \lambda_g^2+\hat a_1 \lambda_g +\hat a_0
\end{equation}
and the $\sigma_\lambda^{\rm cal}$ calibration uncertainties are
\begin{equation}
    \sigma_\lambda^{\rm cal} = \left(\vec J_c^T \mathbf{C}_\lambda \vec J_c\right)^{1/2},\quad \vec J_c = \left(\lambda_g^3, \lambda_g^2, \lambda_g, 1\right).
\end{equation}

In Figure~\ref{fig:spectro_calib_syst}, we plot the residuals of the fit $c(\lambda_g)-\lambda_t$ in the upper panel, showing the agreement between the re-scaled data uncertainties and the uncertainties propagated to the third order polynomial function $c(\lambda_g)$ using $\mathbf{C}_\lambda$. In the lower panel, the spectrograph calibration systematic uncertainties $\sigma_\lambda^{\rm cal}$ are emphasised: they are lower than $\SI{0.1}{\nm}$ in the entire wavelength range, even lower than \SI{0.025}{\nm} in the visible spectrum.

%spectrograph_calibration.py
\begin{figure}[!h]
\centering
\includegraphics[width=\columnwidth]{spectrograph_calibration_syst}
\caption{Spectrograph calibration plot. Top: difference between the tabulated Hg-Ar emission line wavelengths $\lambda_t$ and the wavelengths computed using the spectrograph calibration function $c(\lambda_g)$. Data taken before and after the data acquisition campaign are superimposed. Their uncertainties were re-scaled with a common factor to get a final reduced $\chi^2$ of one. The red band represents the systematic uncertainty band from the $c(\lambda_g)$ fit on data with the uncertainty rescaling. Bottom: emphasis on the systematic uncertainties of the spectrograph calibration procedure.}\label{fig:spectro_calib_syst}
\end{figure}

During the data acquisition campaign, we did not unplug and plug again the optical fibre going from the integrating sphere to the spectrograph. Such an action on the spectrograph side could have changed the wavelength calibration in principle, even if beforehand, we checked the fibre placement in the spectrograph was very reproducible and did not change the wavelength scale.
 
\subsection{Data set description}
\label{sec:cbp_datadesc}

\begin{figure*}[!h]
\centering
%cbp_paper_plots
\includegraphics[width=\columnwidth]{sc_dataset_469}
\includegraphics[width=\columnwidth]{sc_dataset_966}
\caption{Data set examples. From top to bottom: typical data sets for digital analyser (pin state 4 is the Keithley output, pin state 2 the laser trigger output, pin state 1 the Keysight start and end time acquisition time stamps), charges in the photodiode, charges in the solar cell, flux in the spectrograph. Left: typical data set at \SI{469}{\nm}. Right: typical data set at \SI{966}{\nm}.}\label{fig:sc_dataset_examples}
\end{figure*}

A typical dataset to measure the CBP response is the emission of laser bursts in the solar cell at a given wavelength. Charges in the photodiode and solar cell are recorded jointly, along with the flux in the spectrograph and the time stamps in the digital analyser (see examples in Figure~\ref{fig:sc_dataset_examples}).

The laser emits pulses at a fixed rate of \SI{1}{\kilo\hertz}, with a power that highly depends on the wavelength. To ensure that all instruments work in their linear regime, without saturations, and control the dark current, we decided to shoot light in the solar cell in bursts of pulses, separated by dark times at least as long as the burst length. The number of pulses was set to maintain a relatively constant total accumulated charge in the monitoring photodiode, which is the common instrument between the CBP and \SD response measurements and the one receiving the maximum power in the system (Figure~\ref{fig:npulses}). Therefore, we expect to minimise eventual non-linear effects in the instrumental chain. The CBP system's linearity is checked by varying the laser power in Section~\ref{sec:sc_linearity}. However, we know we have a $1/f$ noise in the solar cell instrumental chain. To limit the amplitude of these random fluctuations during the burst, we set the maximum number of pulses to 200 (so \SI{200}{\ms}). If more pulses were requested to maintain the signal level in the photodiode, we compensated with a higher number of bursts.

We used the \SI{5}{\mm} pinhole with the largest laser power mode to get the highest signal-to-noise ratio in the solar cell and the \SD telescope. The solar cell accumulates around \SI{4}{\nano\coulomb} in a burst, with no change of Keysight range. % The Keysight accuracy in this regime is \todo{to check}. 
During a solar cell measurement run, the laser wavelengths range between \SI{350}{\nano\meter} and \SI{1100}{\nano\meter} included with steps of \SI{1}{\nm}, but are randomly chosen to avoid that long-range $ 1/f$ mode in the solar cell dark current correlates neighboured data points of the CBP response. Several runs were accumulated to enhance the signal-to-noise ratio. In particular, five runs were recorded just before the \SD telescope measurement, and five new runs were launched just after.

Runs with different settings have been conducted to estimate systematic uncertainties. We varied the laser global power (QSW) to assess the linearity of the instrumental light and check the ambient light additive contamination. We varied the solar cell distance to estimate the output CBP scattered light.

%cbp_paper_plots.py
\begin{figure}[!h]
\centering
\includegraphics[width=\columnwidth]{npulses}
\caption{Number of laser pulses per burst used for solar cell and StarDice runs.}\label{fig:npulses}
\end{figure}


\subsection{Data reduction}

\todo{check that time stamps are explicitly described for all data sets}

In one data set, we accumulated several laser bursts, and a per-burst analysis was conducted to be able to remove outliers more easily. All bursts are combined only at the end of the analysis to get the CBP and StarDice responses.

\subsubsection{Photodiode data reduction overview}
\label{sec:pd_reduction}

Photodiode data are charge time series with stair-case shape. Each step is a laser burst, and the step height roughly gives the charge $\Qphotmes$ accumulated during a laser burst (see Figures~\ref{fig:sc_dataset_examples} and~\ref{fig:pd_reduc}). The length of the charge rise is the laser burst duration $\tau_b$ while flat sequences are dark times. For the photodiode, the time stamps come directly from the digital analyser clock, with a sampling at \SI{50}{\hertz}. The digital analyser clock also provides the time stamps of the laser pulses. 

For each burst, we fit straight lines in the charge sequence during the dark times before and after a laser burst, removing the closest points to the burst. We call $t_1$ and $t_2$ the time stamps of the beginning and ending of the laser burst, respectively, given by the laser trigger output itself.%\footnote{Technically, the laser time stamps provide the beginning of the pulse, so we add \SI{1}{\ms} to the last trigger time stamps from the laser to account for the pulse duration.}. 
The accumulated charge $\Qphotmes$ during a burst is then
\begin{align}
q_{\rm phot}^{\rm dark}(t)  = & a_{\rm phot} t + b_{\rm phot} \\
\Qphotmes  = & q_{\rm phot,2}^{\rm dark}(t_2) - q_{\rm phot,1}^{\rm dark}(t_1)\\ & - \frac{1}{2} \left[ q_{\rm phot,1}^{\rm dark}(t_2) -  q_{\rm phot,1}^{\rm dark}(t_1) + q_{\rm phot,2}^{\rm dark}(t_2) - q_{\rm phot,2}^{\rm dark}(t_1)  \right]   \notag 
\end{align}
where $q_{\rm phot,j}^{\rm dark}(t_i)$ is the line fit of the dark part $j$ ($j=1$ before the burst, $j=2$ after) evaluated at time $t_i$. Doing so, the subtraction $q_{\rm phot,2}^{\rm dark}(t_2) - q_{\rm phot,1}^{\rm dark}(t_1)$ gives the raw height of the burst step in the charge sequence\footnote{Moreover, doing the subtraction removes systematics inaccuracy coming from the Keithley electrometer.} while the terms in brackets remove the averaged contribution of the dark current using both dark times before and after the burst. Note that we do not model anything during the burst time, as the laser power stability does not guarantee that this can be modelled with a simple mathematical function. The only model assumption is that dark sequences are modelled with straight lines.

The fit of the two parameters $a_{\rm phot}$ and $b_{\rm phot}$ of each $q_{\rm phot,j}^{\rm dark}(t)$ line models is performed via a standard $\chi^2$ minimisation, using the \texttt{curve\_fit} method from python library \texttt{scipy}. Uncertainties on the data points, equally weighted, are tuned so that the final reduced $\chi^2$ is one. Doing so, we assume that the fit residuals are only due to Gaussian instrumental noise, as justified by Figure~\ref{fig:pd_reduc}. The analysis of the residuals also confirms our choice to use order 1 polynomial functions to model the dark sequences.



%cbp_paper_plots.py
\begin{figure}[!h]
\centering
\includegraphics[width=\columnwidth]{pd_reduc_966}
\caption{Photodiode charge sequence reduction process at wavelength $\lambda_L=\SI{469}{\nm}$. Vertical blue lines indicate the laser starts and stops, while black lines flank the dark sequences. Coloured horizontal lines are fitted during dark times. The top panel shows the raw charges acquired with the photodiode, while the bottom panel shows the residuals of the linear fits during dark times.}\label{fig:pd_reduc}
\end{figure}

Covariance matrix uncertainties from all linear model parameters are then propagated to compute the statistical uncertainty $\Sphotstat$ of $\Qphotmes$ per burst. They are typically of the order of the residual RMS, around \SI{5e-5}{\nano\coulomb} (Figure~\ref{fig:pd_reduc}), more than three orders of magnitude below the typical $\Qphotmes$ values. We tested the fitting procedure on pure dark sequences and found an unbiased null measurement $\Qphot^{\rm dark}$ with a pull distribution of RMS $\;\approx 1$ whatever the burst duration $\tau_b$: computed statistical uncertainties $\Sphotstat$ nicely covers the data Gaussian noise (Figure~\ref{fig:charge_pull}).

%stardice_analysis/keithley_dark_and_bias.py
\begin{figure}[!h]
\centering
\includegraphics[width=\columnwidth]{pd_sc_stat_pull}
\caption{Top: pull distributions for photodiode charge measurements during pure dark time series $\Qphot^{\rm dark} / \Sphotstat$, with different burst durations $\tau_b$. Cyan curve represents a Gaussian distribution of mean 0 and RMS 1. Bottom: same but for the solar cell case $\Qsolar^{\rm dark} / \Ssolarstat$.}\label{fig:charge_pull}
\end{figure}



\subsubsection{Solar cell data reduction overview}
\label{sec:solar_reduction}

Solar cell charge time series are very similar to photodiode time series (see Figures~\ref{fig:sc_dataset_examples} and~\ref{fig:sc_reduc}). However, they are affected by two supplementary contributions as seen in the noise power spectrum: a random $1/f$ noise and power line harmonics mainly at \SI{50}{\hertz} and \SI{100}{\hertz} (Figure~\ref{fig:darkcurrentspectrum}). Timestamps come directly from the Keysight electrometer. However, as this device sends triggers at the start and end of the acquisition, the electrometer clock is re-scaled using the digital analyser. Doing so, all electrometers are synchronised via the digital analyser's internal clock.


%cbp_paper_plots.py
\begin{figure}[!h]
\centering
\includegraphics[width=\columnwidth]{sc_reduc_966}
\caption{Solar cell charge sequence reduction process at wavelength $\lambda_L=\SI{469}{\nm}$. Pairs of vertical black at the left and right of a burst encompass the fitted dark sequences. Coloured curves are the dark model fitted within these dark times. The top panel shows the raw charges acquired with the solar cell, while the bottom panel shows the residuals to the dark model fit during dark times only.}\label{fig:sc_reduc}
\end{figure}

As for the photodiode, we modelled the dark sequence before and after each burst. The solar cell dark model is the sum of a linear function plus two sinus functions at fixed frequency \SI{50}{\hertz} and \SI{100}{\hertz}:
\begin{align}
    q_{\rm solar}^{\rm dark}(t) = a_{\rm solar}t + b_{\rm solar} & + A_{50} \sin \left( 100 \pi t + \phi_{50}\right)  \notag \\  & +  A_{100} \sin \left( 200 \pi t + \phi_{100}\right)
\end{align}
where $a_{\rm solar}, b_{\rm solar}, A_{50}, A_{100}, \phi_{50}$ and $\phi_{100}$ are free parameters fitted on data. 
Again, we call $t_1$ and $t_2$ the time stamps of the beginning and end of the laser burst, respectively, given by the laser trigger output itself.
The accumulated charge $\Qsolarmes$ during a burst is then
\begin{align}\label{eq:qsolar}
\Qsolarmes  = & q_{\rm solar,2}^{\rm dark}(t_2) - q_{\rm solar, 1}^{\rm dark}(t_1) \\  &  - \frac{1}{2} \left[q_{\rm solar,1}^{\rm dark}(t_2) - q_{\rm solar,1}^{\rm dark}(t_1) + q_{\rm solar,2}^{\rm dark}(t_2) - q_{\rm solar,2}^{\rm dark}(t_1)  \right]    \notag
\end{align}
where the indices $1$ and $2$ refer again to data before and after the burst, respectively. The purpose of the term in brackets is to subtract an estimation of the dark current contribution during the burst itself. Free $q_{\rm solar}^{\rm dark}(t)$ parameters were fitted, minimising a $\chi^2$ with a Newton-Raphson gradient descent. Power lines are well fitted by the model (no more periodic oscillations in the residuals) as shown in Figure~\ref{fig:sc_reduc_zoom} where $A_{50}$ was about $\SI{50}{\pico\coulomb}$.
The $1/f$ noise is not modelled in $q_{\rm solar}^{\rm dark}(t)$ as it is subdominant, but the lowest frequency modes are captured by the values of $a_{\rm solar}$ and $b_{\rm solar}$. To get a close estimate of their contributions during the burst, we fit the dark sequences only during $\tau_b/2$ around the burst to rely on their extrapolation inside the burst window (with a minimum of 12 data points to encompass at least one \SI{50}{\hertz} period). %Doing so, both linear functions are the sum of both darks containing the same $1/f$ noise power as during the laser burst and only the long-wave modes that pollute the laser burst.
Indeed, as exhibited in Figure~\ref{fig:sc_reduc_zoom}, fits of $q_{\rm solar,2}^{\rm dark}(t)$ after the first burst (orange) and of $q_{\rm solar,1}^{\rm dark}(t)$ before the second burst (green) do not superimpose because of the long-range $1/f$ noise that deviates the data from a global linear relationship. So focusing the fits in two windows of size $\max(\SI{22}{\ms},\tau_b/2)$ permits to capture contaminating $1/f$ modes no longer than $\tau_b$. 

Charge value $\Qsolarmes$ is then computed for each laser burst following Equation~\ref{eq:qsolar}. Concerning estimating its statistical uncertainties $\Ssolarstat$, we add in quadrature the contributions from the parameter covariance matrix and the RMS of the residuals (to account for uncaptured $1/f$ modes). We trained the fitting procedure on pure dark data and noticed that $\Ssolarstat$ was too small to account for $\Qsolar^{\rm dark}$ null measurement dispersion. The RMS of the pull distributions was linearly dependent on the burst duration $\tau_b$, showing $\Ssolarstat$ did not capture all the long-range $1/f$ noise. Therefore, we corrected all $\Ssolarstat$ values with a multiplicative factor dependent on $\tau_b$. Doing so, we found an unbiased null measurement $\Qsolar^{\rm dark}$ with a pull distribution of RMS $\;\approx 1$ whatever the burst duration $\tau_b$ (Figure~\ref{fig:charge_pull}). 

%cbp_paper_plots.py
\begin{figure}[!h]
\centering
\includegraphics[width=\columnwidth]{sc_reduc_966_zoom}
\caption{Same as Figure~\ref{fig:sc_reduc} but zoomed on the second dark sequence. The orange model is fitted on dark data on the left of the plot, while the green model is fitted on dark data on the right.}\label{fig:sc_reduc_zoom}
\end{figure}


\subsubsection{Spectrograph data reduction overview}
\label{sec:spectro_reduction}

The spectrograph exposure times were adjusted to avoid the saturation level for all wavelengths. However, the sampling rate was not tunable, and time stamps were unavailable. So we took as many spectra $N(\lambda_p)$ as possible during an acquisition and then analysed them. Some contained laser light, and others were darks. We assumed a spectrum is dark if there are no spectrograph pixels above its median plus a threshold in the region around the laser wavelength. This algorithm works for most spectra except for very weak laser lines that come too close to a hot pixel, but very rarely.

After this sorting of the spectra in two categories ("laser on" and "laser off"), we subtract the master bias $B(\lambda_p)$ and compute a master dark spectrum $D(\lambda_p)$, taking the median of all the dark spectra. This master dark is subtracted from all spectra. This removes the spectrograph baseline and hot pixels. The spectra containing laser lines are stacked to increase the signal-to-noise ratio. Let's call $S(\lambda_p)$ the stacked spectrum after bias and dark subtraction:
\begin{align}
D(\lambda_p) & = \sum_{i \in \left\lbrace \text{laser off}\right\rbrace}\left( N_i(\lambda_p) - B(\lambda_p)\right), \\
    S(\lambda_p) & = \sum_{i \in \left\lbrace \text{laser on}\right\rbrace}\left( N_i(\lambda_p) - D(\lambda_p) - B(\lambda_p)\right).
\end{align}
The spectrograph error model is applied to get the intensity uncertainties $\sigma_p$ for each spectrum $i$:
\begin{equation}\label{eq:spectro_error_model_data}
\sigma^2_{i,p} =\sigma_{ro}^2 +  (N_i(\lambda_p) - B(\lambda_p))/G + \sigma_G^2 (N_i(\lambda_p) - B(\lambda_p))^2
\end{equation}
As the signal-to-noise in the stacked spectrum is very high, we assimilate the $S(\lambda_p)$ value to its empirical average to get $\sigma_{i,p}$. The uncertainties are propagated to the stacked spectrum $S(\lambda_p)$.

Then, to detect emission lines in $S(\lambda_p)$, we fit locally Gaussian profiles on top of a linear background. The $\lambda_g$ fit is unweighted as for the spectrograph calibration process. We searched for the laser line and lines at \SI{532}{\nm} and from two-photon conversions. Indeed, we noticed the presence of a \SI{532}{\nm} line in the regime 532 to \SI{669}{\nm} and of a weak line when the laser is set to wavelength above \SI{1064}{\nm}. If we call $\lambda_L$ the wavelength at which the laser was set, we observed the production of photons at wavelength $\lambda_{\text{comp}}$, which seems given by
\begin{equation}
 \frac{2}{\SI{1064}{\nm}} \approx \frac{1}{\lambda_L} + \frac{1}{\lambda_{\text{comp}}}
 \end{equation} 
due to the conversion of two \SI{1064}{\nm} photons into a laser photon at $\lambda_L$ and a complementary photon at $\lambda_{\text{comp}}$ that ended in the laser beam. In practice, we observed a small emission line in the spectrograph when $\lambda_L > \SI{1064}{\nm}$, nearly symmetrical of the laser line with respect to \SI{1064}{\nm}. 

An example of a stacked spectra with the detection of the laser line at \SI{643}{\nm} and the pump line at \SI{532}{\nm} is shown in Figure~\ref{fig:spectro_reduc_643}, before spectrograph wavelength calibration. The average intensity value far from the emission lines is zero, with statistical error bars compatible with the observed dispersion. Intensity uncertainties are propagated up to the line centroid wavelength. 

%cbp_paper_plots.py
\begin{figure}[!h]
\centering
\includegraphics[width=\columnwidth]{spectro_reduc_643}
\caption{Fit of Gaussian profiles (blue lines) in the stacked spectra (red dots) for a laser line set at $\lambda_L=\SI{643}{\nm}$, with raw spectrograph wavelengths on the abscissa axis. The contamination line at \SI{532}{nm} is also visible. The black vertical line gives the fitted laser line centroid. The grey vertical lines delimit a region of $\pm 3\sigma_g$ around the centroid, where $\sigma_g$ is the fitted RMS of the Gaussian profile. The total laser flux is the sum of the pixel values in this region.}\label{fig:spectro_reduc_643}
\end{figure}

We used the \SI{532}{\nm} line to check the quoted statistical errors $\sigma_\lambda^{\rm noise}$ on wavelength. For all data sets, we plotted the fitted line position versus time. The RMS is compatible with the quoted error bars on wavelength for lower signal-to-noise ratio data sets. But for the high signal-to-noise ratio data sets, some dispersion is unaccounted, explained by the fact that the Gaussian profile is an incomplete model of high signal-to-noise lines. Therefore, we added a $\sigma_\lambda^{\rm PSF}=\SI{0.012}{\nm}$ statistical uncertainty due to PSF modelling on wavelength uncertainty to get a normal distribution for the residuals normalised by the full statistical uncertainties (see Figure~\ref{fig:wavelength_error_model_consistency}), the latter being
\begin{equation}
    \left(\sigma_{\lambda}^{\rm stat}\right)^2 =  \left(\sigma_{\lambda}^{\rm noise}\right)^2 +  \left(\sigma_{\lambda}^{\rm PSF}\right)^2.
\end{equation}

%cbp_paper_plots.py
\begin{figure}[!h]
\centering
\includegraphics[width=\columnwidth]{wavelength_error_model_consistency.png}
\caption{Top: all measured \SI{532}{\nm} pump line calibrated wavelengths with respect to exposure index ($\approx\num{250000}$ data points) for the four solar cell runs. Bottom: distributions of residuals to the mean wavelength normalized by $\sigma_{\lambda}^{\rm stat}$ (colored bars) with RMS quoted in legend. A normal distribution of RMS 1 is overplotted for comparison.}\label{fig:wavelength_error_model_consistency}
\end{figure}

Finally, the spectrograph wavelength calibration is applied to detected wavelengths, and corresponding systematic uncertainty is added. The final quoted wavelength is $\lambda_c$ and the final wavelength uncertainty is
\begin{equation}
  \left(\sigma_{\lambda}\right)^2 =  \left(\sigma_{\lambda}^{\rm stat}\right)^2 +  \left(\sigma_{\lambda}^{\rm cal}\right)^2.   
\end{equation}
The composition of the final wavelength error budget is detailed in Section~\ref{sec:wavelength_syst}.

The laser line flux $\Qspectromain$, the \SI{532}{\nm} line flux $\Qspectro^{532}$ and the $\lambda_{\rm comp}$ line flux $\Qspectro^{\rm comp}$ flux are computed summing the pixels in a window of $\pm 3 \sigma_g$ where $\sigma_g$ is the RMS of the fitted Gaussian profile, after dark and bias subtraction. Associated statistical uncertainties are computed from the standard error propagation of the $\sigma_{i,p}$ values. These fluxes are used to correct solar cell and photodiode charges as well as the \SD photometry from the $\SI{532}{\nm}$ line contamination (see Section~\ref{sec:532_cont}).  %As the \SI{532}{\nm} line flux is always at most 1\% of the laser line flux,

\subsection{CBP ratio of charges}

A first CBP response estimation can be computed as $r_{\rm CBP}^{\rm mes} = \Qsolarmes/\Qphotmes$ to check the statistical uncertainties and then analyse systematic uncertainties. The ratio of charges is presented in Figure~\ref{fig:cbp_charge_ratio}. Each black point is a ratio of charge measurements from one burst at one wavelength $\lambda_L$. They follow a smooth curve in $\lambda_L$. The $r_{\rm CBP}^{\rm mes}$ statistical uncertainties 
\begin{equation}
    \sigma_{\mathrm CBP}^{\rm stat} = r_{\rm CBP}^{\rm mes} \sqrt{\left(\frac{\Ssolarstat}{\Qsolarmes}\right)^2 +  \left(\frac{\Sphotstat}{\Qphotmes}\right)^2 }
\end{equation}
are higher in the blue part as the laser is weaker in this regime (around $0.1\%$) than in the red part (around $0.01\%$). They were correctly estimated as the difference between all data points and a spline curve passing through the points, normalised by the statistical uncertainties, follows a Gaussian distribution of mean 0 and RMS$\;\approx 1$, as expected. If we compute a mean CBP response

%cbp_paper_plots.py
\begin{figure}[!h]
\centering
\includegraphics[width=\columnwidth]{fig/cbp_charge_ratio.png}
\caption{CBP charge ratio $\Qsolarmes/\Qphotmes$ as a function of $\lambda_L$. Top: every black point is a charge measurement ratio from one burst, the blue curve is the $5\sigma$-clipped average. Middle: relative uncertainties on $\Qsolarmes/\Qphotmes$. Bottom: pull distribution after spline subtraction.}\label{fig:cbp_charge_ratio}
\end{figure}


\subsection{Systematics}



\subsubsection{Wavelength calibration}\label{sec:wavelength_syst}

%cbp_paper_plots.py
\begin{figure}[!h]
\centering
\includegraphics[width=\columnwidth]{wavelength_stability.png}
\caption{Difference between the calibrated $\lambda_c$ and requested laser wavelengths $\lambda_L$, for all the $\approx \num{350000}$ laser line wavelengths acquired during our measurement campaign.}\label{fig:wavelength_stability}
\end{figure}


The realised wavelength is never the one a priori asked, as shown in Figures~\ref{fig:sc_dataset_examples} and~\ref{fig:wavelength_stability}. However, we observed remarkable repeatability of the correspondence between the set wavelength $\lambda_L$ and the realised wavelength $\lambda_g$ in Figure~\ref{fig:wavelength_stability}. This figure represents every $\approx\num{350000}$ laser bursts shot and analysed during the CBP measurement campaign. The superimposition of all measurements at each laser wavelength $\lambda_L$ exhibits good stability of the laser with time and of the $\lambda_g$ wavelength fit with line intensity. The RMS is better than \SI{0.02}{\nm} at most wavelengths, except where line determination is ambiguous: close to the contamination lines at $\SI{532}{nm}$ and $\SI{1064}{nm}$, and close to the sensor edges. As the bijection between $\lambda_L$ laser wavelengths and realised wavelengths $\lambda_c$ is unambiguous, in some figures of this paper, we use the set laser wavelength $\lambda_L$ for clarity. 

Figure~\ref{fig:wavelength_error_budget} details the total error budget from wavelength calibration for three different runs: two runs shooting at StarDice with two pinholes and one run shooting at the solar cell. Uncertainty on wavelength $\sigma_\lambda$ is primarily dominated by wavelength calibration uncertainties in the range 400 to \SI{1080}{\nm}. Statistical uncertainty dominates around $\SI{532}{\nm}$ and close to the spectrograph sensor edges. Except in these cases, $\sigma_\lambda$ is well below $\SI{1}{\angstrom}$, which allows for precisely measuring the telescope filter band-passes better than the Angstrom level.   

\todo{Expliquer comment on propage les incertitudes de $\sigma_\lambda$ sur les résultats finaux.}

%cbp_paper_plots.py
\begin{figure}[!h]
\centering
\includegraphics[width=\columnwidth]{spectrograph_error_budget.png}
\caption{Total error budget (red) of calibrated wavelengths $\lambda_c$ for three different runs (from top to bottom: StarDice run with \SI{75}{\um} pinhole, solar cell run with \SI{5}{mm} pinhole, StarDice run with \SI{5}{mm} pinhole) as a function of the laser set wavelength $\lambda_L$. Detection uncertainties (grey) represent PSF modelling uncertainties and Gaussian fit uncertainties, while calibration uncertainties (black) come from the Hg-Ar lamp calibration procedure. }\label{fig:wavelength_error_budget}
\end{figure}

To evaluate systematic uncertainty on the CBP response due to the wavelength calibration, we computed $r_{\rm CBP}$ at $\lambda_c+\sigma_{\lambda^{\rm cal}}$ and $\lambda_c-\sigma_{\lambda^{\rm cal}}$. The difference between both CBP responses is well below the statistical uncertainty, as the CBP response varies slowly in $\lambda$ whole $\sigma_\lambda$ is sub-Angstrom (see Figure~\ref{fig:cbp_charge_ratio}). The importance of $\sigma_\lambda$ as a systematic uncertainty lies essentially in the determination of the telescope filter band-passes and the position of their sharp edges.


\subsubsection{Ambient light}\label{sec:sc_linearity}

 %cbp_paper_plots.ipynb
\begin{figure}[h]
    \centering
    \includegraphics[width=\columnwidth]{fig/sc_dark_qswMAX.png}
    \caption{CBP charge ratio $\Qsolar^{\rm dark}/\Qphot^{\rm dark}$ as a function of $\lambda_L$: every black point is a charge measurement ratio from one burst, the blue curve is the $5\sigma$-clipped average.}
    \label{fig:sc_dark}
\end{figure}

To check contamination by ambient light, we ran a solar cell acquisition with a cap on the CBP telescope. In that way, the solar cell can collect indirect light from the laser box when firing. The measurement of this indirect ambient light $\Qsolar^{\rm dark}$ constitutes a "dark" for our CBP calibration. From this specific run, we built a dark CBP response $r_{\mathrm{CBP}}^{\rm dark} = \Qsolar^{\rm dark} / \Qphot^{\rm dark}$ where $\Qsolar^{\rm dark}$ (resp. $\Qphot^{\rm dark}$) is the burst charge collected in the solar cell (resp. the photodiode). For each $\lambda_L$, burst ratios are averaged with a $5\sigma$ clipping, giving the blue curve $r_{\mathrm{CBP}}^{\rm dark}(\lambda_L)$ in Figure~\ref{fig:sc_dark}. The dark contribution in our solar cell data is then evaluated as follows:
\begin{equation}
    \Qsolar^{\rm dark} = r_{\mathrm{CBP}}^{\rm dark}(\lambda_L) \times \Qphotmes(\lambda_L)
\end{equation}
and subtracted from all our measurements. This correction is the main contribution to instrumental non-linearity we identified. 


\subsubsection{Laser light contamination}
\label{sec:532_cont}

As described in Section~\ref{sec:cbp}, the light source used is a tunable laser using a pump laser at \SI{532}{\nano\meter}, which has different regimes. When we operated within the range [532 - 644] nm, in the spectrograph, we observed a contamination light at \SI{532}{\nano\meter} for all wavelengths in this range. 

We must account for this light contamination to get the true amount of charges $\Qsolarcal$ coming from main laser line. Therefore, we built a model for the \SI{532}{\nano\meter} contribution observed in the range [532 - 644] nm. The total charge measured in the solar cell $\Qsolarmes$ is the sum of the charges from the main wavelength $\lambda_L$ and the charges from contaminations, like the \SI{532}{\nm} contamination. This is the same for the total charges measured in the photodiode $\Qphotmes$ with $\Qphot$ and $\Qphot^{532}$ respectively the charges from the main laser line and the charges from the \SI{532}{\nm} contamination:
\begin{align}
%\Qsolarmes(\lambda_L) & = \Qsolar(\lambda_L) + \Qsolar^{532}(\lambda_L) + r_{\mathrm{CBP}}^{\rm dark}(\lambda_L) \times \Qphotmes(\lambda_L) \label{eq:qsolar_mes} \\
\Qphotmes(\lambda_L) & = \Qphot(\lambda_L) + \Qphot^{532}(\lambda_L) \label{eq:qphot_mes}
\end{align}

In the spectrograph, we measured two fluxes from two separated peaks: the one from the main wavelength $\Qspectromain$, and the one from the \SI{532}{\nm} contamination $\Qspectro^{532}$. As the light is homogeneous in the integrating sphere, the proportion of contamination light and laser light is the same at the entrance of both instruments. This translates into the equality
\begin{equation}
    \frac{\Qspectro^{532}}{\Qspectromain} \times \frac{\Espectro(\lambda_L)}{\Espectro(532)} = \frac{\Qphot^{532}}{\Qphot} \times \frac{\Ephot(\lambda_L)}{\Ephot(532)}
    \label{eq:prev_alpha}
\end{equation}
where $\Ephot(\lambda_L)$ and $\Espectro(\lambda_L)$ are the quantum efficiencies of the photodiode and the spectrograph sensor with its optical fiber, respectively. Their values can be taken from manufacturer data sheets (Figure~\ref{fig:QEs}), or their ratio
\begin{equation}\label{eq:eta}
\eta(\lambda) = \frac{\Espectro(\lambda)}{\Ephot(\lambda)} = \frac{\Qspectromain(\lambda)}{\Qphot(\lambda)}
\end{equation}
can be measured directly with all our measurements in spectrograph ADU per photodiode unit (Figure~\ref{fig:QEs}). In the following, we used the estimate of $\eta(\lambda)$ instead of the vendor curves, with an uncertainty given by its RMS around a smooth spline curve.

Then, we defined the $\alpha(\lambda)$ ratio function as
\begin{equation}
    \alpha(\lambda_L) = \frac{\Qphot^{532}(\lambda_L)}{\Qphot(\lambda_L)} = \frac{\Qspectro^{532}(\lambda_L)}{\Qspectromain(\lambda_L)} \times \frac{\Espectro(\lambda_L)\Ephot(532)}{\Ephot(\lambda_L)\Espectro(532)} 
    \label{eq:alpha}
\end{equation}
to compute the level of contamination in the photodiode $\Qphot^{\lambda_L}/\Qphot^{532}$.

%cbp_paper_plots.py
\begin{figure}[h]
    \centering
    \includegraphics[width=\columnwidth]{fig/qe_phototiode_spectro.pdf}
    \caption{Quantum efficiencies of the spectrograph (red) and the photodiode (cyan), with the $\eta(\lambda)$ ratio (black).}
    \label{fig:QEs}
\end{figure}
    
We measured the $\alpha(\lambda_L)$ ratio with the right-hand side of Equation~\ref{eq:alpha} at every wavelength with the spectrograph data (Figure~\ref{fig:alpha_532}). We accumulated all available data with $\sigma_c < \SI{0.1}{\nm}$ (to reject data sets with confusion with the \SI{532}{\nm} line and the main laser line). We fitted a first degree polynomial function on all the $\Qspectro^{532}(\lambda_L)$ data points we had to allow us for a reasonable extrapolation in the range 532 - \SI{540}{\nm} where unambiguous data are missing. Uncertainties on  $\Qspectro^{532}$. Then, we computed the ratio $\alpha(\lambda_L)$ with that model for $\Qspectro^{532}$. It gives a rather flat curve between 540 and \SI{644}{\nm} of about 1\%, with smaller wiggles. Uncertainties are propagated in the extrapolation range using the standard error propagation formula, taking into account the covariances between the 2 model parameters. With this $\alpha$ model, we deduced the true photodiode charges coming from the main laser line at $\lambda_L$
\begin{equation}
        \Qphotcal(\lambda_L) \equiv  \frac{\Qphotmes(\lambda_L)}{1 + \alpha(\lambda_L)} \quad\text{if}\ \lambda_L \in \left[532, 644\right]\mathrm{\,nm}\label{eq:qphot_cal532}
\end{equation}
We introduce here the notation $\Qphotcal$ as the final calibrated amount of charges detected in the photodiode per laser burst. 

Light detected by the solar cell has gone through the CBP optics, with a response $\Rcbp$.
Then, the contribution of the \SI{532}{\nano\meter} photons collected by the solar cell is computed as
\begin{equation}
\begin{aligned}
    \Qsolar^{532}(\lambda_L) & = \Rcbp(532)  \Qphot^{532}(\lambda_L) \\ 
    & = \Rcbp(532) \frac{ \alpha(\lambda_L) }{1+ \alpha(\lambda_L)} \Qphotmes(\lambda_L).
    \label{eq:qsolar_cal532}
\end{aligned}
\end{equation}
Thanks to the spectrograph data, using Equations~\ref{eq:qphot_cal532} and~\ref{eq:qsolar_cal532}, we can correct all our measurements from the \SI{532}{\nano\meter} in both the photodiode and the solar cell. The impact of this correction is illustrated in Section~\ref{sec:cbp_summary}.


\begin{figure}[h]
    \centering
    \includegraphics[width=\columnwidth]{fig/alpha_532_qswMAX.pdf}
    \caption{Top: all available data points for $\alpha$ computed from the spectrograph measurements (blue crosses) and degree 9 polynomial fit (orange) with its uncertainty (shaded orange band). Bottom: difference between data and model.}
    \label{fig:alpha_532}
    %~/stardice/analysis/cbp_paper/golden_sample_analysis/dr3/532nm_correction.ipynb
\end{figure}


We did the same correction for the complementary wavelength line $\lambda_{\rm comp}$ appearing after $\lambda_L > \SI{1064}{\nm}$. The correction coefficient $\beta$ analogue to $\alpha$ is
\begin{equation}
    \beta(\lambda_L) = \frac{\Qphot^{\lambda_{\rm comp}}(\lambda_L)}{\Qphot(\lambda_L)} = \frac{\Qspectro^{\lambda_{\rm comp}}(\lambda_L)}{\Qspectromain(\lambda_L)} \times \frac{\Espectro(\lambda_L)\Ephot\contcomp}{\Ephot(\lambda_L)\Espectro\contcomp} 
    \label{eq:beta}
\end{equation}
and is represented Figure~\ref{fig:beta}. The $\Qspectro^{\lambda_{\rm comp}}$ data points were modelled by a degree 1 polynomial function to allow us extrapolating $\beta$ in the range [1064 - 1070]\,nm. Similarly, the calibrated amount of charges in the photodiode corrected from the $\lambda_{\rm comp}$ photons is
\begin{equation}
        \Qphotcal(\lambda_L) \equiv  \frac{\Qphotmes(\lambda_L)}{1 + \beta(\lambda_L)} \quad\text{if}\ \lambda_L > \SI{1064}{\nano\meter}
        \label{eq:qphot_cal1064}
\end{equation}
and their contribution $\Qsolar^{\lambda_{\rm comp}}$ in the solar cell is given by
\begin{equation}
\begin{aligned}
    \Qsolar^{\lambda_{\rm comp}}(\lambda_L) & = \Rcbp(\lambda_{\rm comp})  \Qphot^{\lambda_{\rm comp}}(\lambda_L) \\ 
    & = \Rcbp(\lambda_{\rm comp}) \frac{ \beta(\lambda_L) }{1+ \beta(\lambda_L)} \Qphotmes(\lambda_L).
    \label{eq:qsolar_cal1064}
\end{aligned}
\end{equation}

Both corrections' impact is illustrated in Section~\ref{sec:sc_linearity} and Figure~\ref{fig:SCqswlinearity}, and Section~\ref{sec:sd_contaminations}.


\begin{figure}[h]
    \centering
    \includegraphics[width=\columnwidth]{fig/beta_532_qswMAX.pdf}
    \caption{Same as Figure~\ref{fig:alpha_532} but for $\beta$ correction coefficient.}
    \label{fig:beta}
    %~/stardice/analysis/cbp_paper/golden_sample_analysis/dr3/1064nm_correction.ipynb
\end{figure}


\subsubsection{Integrating sphere fluorescence}\label{sec:fluorescence}


\begin{figure}[h]
    \centering
    \includegraphics[width=\columnwidth]{fig/spectro_stack_fluo_model.pdf}
    \includegraphics[width=\columnwidth]{fig/QPDfluo_model.pdf}
    \caption{Top: integrating sphere fluorescence spectrum from a stack of all solar cell run data with $\lambda_L < \SI{370}{\nano\meter}$ (blue crosses), with the best-fit model (orange line). Bottom: estimated fluorescence contribution in photodiode measured charges $\Qphot^{\rm fluo}$ as a function of wavelength (blue crosses) with a fitted third-order polynomial function (orange line).}
    \label{fig:fluo}
    %~/stardice/analysis/cbp_paper/golden_sample_analysis/dr3/1064nm_correction.ipynb
\end{figure}

Our integrating sphere appeared to be fluorescent at laser wavelengths below \SI{400}{\nano\meter}. The fluorescence of integrating spheres is studied in~\cite{shaw2007ultraviolet}. The fluorescent signal is visible in the \SD camera using the g filter or the grating but is very weak in the spectrograph. To visualise and model it, we stacked all our spectra per bins of \SI{5}{\nano\meter} in $\lambda_L$ (Figure~\ref{fig:fluo} top). The fluorescence signal spans a range of wavelength between $\approx \SI{400}{\nano\meter}$ and $\approx\SI{500}{\nano\meter}$, with an emission peak around with a emission peak at \SI{450}{\nano\meter}. %The peak amplitude depends on the laser wavelength, as expected for a fluorescence phenomenon. 

To estimate the contamination from fluorescence photons, we fitted a fluorescence spectrum model taken from Figure~9 of \cite{shaw2007ultraviolet}, with a constant background and a Moffat profile for the laser line for each stacked spectra. The fluorescence spectrograph flux is converted into photodiode charges $\Qphot^{\mathrm{fluo}}$ using the $\eta(\lambda)$ conversion factor and normalised by the total number of laser pulses (Figure~\ref{fig:fluo} bottom)\footnote{Contrary to the \SI{532}{\nano\meter} line contamination correction, we can not normalise by the flux in the main laser line as it is often noise-dominated in un-stacked spectra when $\lambda_L < \SI{400}{\nano\meter}$.}. We observed that the fluorescence spectrum cancels at $\lambda_L \geq \SI{400}{\nano\meter}$. So, for every wavelength $\lambda_L < \SI{400}{\nano\meter}$, we evaluate and subtract the contribution from the fluorescence contamination in the photodiode using the $\Qphot^{\mathrm{fluo}}(\lambda_L)$ model from Figure~\ref{fig:fluo}:
\begin{equation}
        \Qphotcal(\lambda_L) \equiv  \Qphotmes(\lambda_L) - \Qphot^{\mathrm{fluo}}(\lambda_L) \quad\text{if}\ \lambda_L < \SI{400}{\nano\meter}
        \label{eq:qphot_calfluo}
\end{equation}
We perform identically for $\Qsolarmes$ multiplying by the CBP response at \SI{450}{\nano\meter}: 
\begin{equation}
\begin{aligned}
    \Qsolar^{\rm fluo}(\lambda_L) & = \Rcbp(450)  \Qphot^{\rm fluo}(\lambda_L)
    \label{eq:qsolar_calfluo}
\end{aligned}
\end{equation}

This correction's impact is illustrated later in Section~\ref{sec:sd_contaminations}.

After the fluorescence, \SI{532}{\nano\meter} and $\lambda_{\mathrm{comp}}$ corrections, we updated our $\eta(\lambda)$ estimate and iterated several times to refine the light contamination subtractions.


\subsubsection{Instrumental chain linearity check}\label{sec:sc_linearity}
For the two solar cell runs we undertook, we varied the laser output power by a factor of around 2, namely, QSW at maximum and QSW set at 298. The ratio of the two CBP charge ratios before and after dark subtraction is presented in Figure~\ref{fig:SCqswlinearity}. Basically, no corrections lead to a ~5 permil deviations of the two CBP responses with respect to wavelength. Both CBP responses agree at $\approx 0.5\,$permil for wavelengths above \SI{669}{\nano\meter} when applying dark subtraction. Then, laser contamination correction makes the two CBP responses agree in the [532, 669]\,nm range better than 0.1 permil.

To assess the value of systematic uncertainties on the CBP response due to non-linearities, we take the absolute distance of the binned ratio to unity in four different ranges of wavelengths after dark subtraction and laser contamination correction (red segments in the bottom plot of Figure~\ref{fig:SCqswlinearity}).

%cbp_paper_plots.ipynb
\begin{figure}[h]
    \centering
    \includegraphics[width=\columnwidth]{fig/sc_qsw_ratios.pdf}
    \caption{Ratios of the CBP charge ratio for two different QSW values as a function of $\lambda_L$ coming from the 2022/03/04 solar cell run. Blue is for raw data while red is used for data corrected by dark contribution and laser contaminations. Black dashed lines encompass the permil precision region. Top: ratios for each $\lambda_L$. Bottom: binned ratio for four different wavelength ranges. A similar plot is obtained for the 2022/03/06 solar cell run.}
    \label{fig:SCqswlinearity}    
\end{figure}


\subsubsection{CBP scattered light varying the solar cell distance}

To measure the influence of scattered light in the CBP beam, we measured the CBP throughput by putting the solar cell \SI{16}{\cm} farther and compared it to the initial value. At this new position, we measured again the CBP dark from solar cell $\Qsolar^{\rm dark}$. Dark subtraction and laser contamination correction are applied. The comparison of both transmissions is presented in Figure~\ref{fig:sc_distance}. There is a decrease of the total light of about 3\textperthousand\ with a chromatic effect of about 2.5\textperthousand\ difference between \SI{350}{\nano\meter} and \SI{1100}{\nano\meter}. This constitutes the dominant systematics in the CBP throughput measurement.

%cbp_paper_plots.ipynb
\begin{figure}[h]
    \centering
    \includegraphics[width=\columnwidth]{fig/sc_distance.pdf}
    \caption{Ratios of the CBP charge ratio for two different distances to the solar cell as a function of $\lambda_L$ coming from the 2022/03/08 solar cell run. The long-distance is \SI{16}{\cm} larger than the short distance. Black points are the binned ratio for each $\lambda_L$, and the blue line is a fit whose equation is in legend.}
    \label{fig:sc_distance}
\end{figure}

\subsubsection{Solar cell QE variations with angle and temperature}

\todo{Do we write here the results ?}


\subsubsection{Repeatability}

Finally, we measured three times the value of the CBP response during our measurement campaign. For run $i$, we computed the CBP charge ratio $r_{\rm CBP}^{\mathrm{run}\ i}$, applying dark subtraction and laser contamination correction, binned in \SI{1}{\nano\meter} intervals in $\lambda_L$. Then, we computed the mean CBP response $\overline{r_{\rm CBP}}$ as the mean of the three different runs. We observed $\approx 1$\,permil differences between the three CBP charge ratios and $\overline{r_{\rm CBP}}$ (Figure~\ref{fig:SCrepeatability}), depending slightly with wavelength.


To assess the value of systematic uncertainties on the CBP response due to its stability, we take the maximum absolute distance of the binned ratio to unity in four different ranges of wavelengths (segments the farther from 1 in the bottom plot of Figure~\ref{fig:SCrepeatability}). The strongest systematics comes from the scattered light.

%cbp_paper_plots.ipynb
\begin{figure}[h]
    \centering
    \includegraphics[width=\columnwidth]{fig/sc_runi_ratios.png}
    \caption{Ratios of the CBP charge ratio $r_{\rm CBP}^{\mathrm{run}\ i}\;/\;\overline{r_{\rm CBP}}$ for three different runs as a function of $\lambda_L$. Black dashed lines encompass the per-mil precision region. Top: ratios for each $\lambda_L$. Bottom: binned ratio for four different wavelength ranges.}
    \label{fig:SCrepeatability}
\end{figure}

\subsubsection{Summary}\label{sec:cbp_summary}

In summary, we defined the calibrated amount of charges in the solar cell as:
\begin{equation}
\Qsolarcal \equiv \Qsolarmes - \Qsolar^{\rm dark} - \Qsolar^{532} - \Qsolar^{\lambda_{\rm comp}} - \Qsolar^{\rm fluo}
\end{equation}
and in the photodiode as:
\begin{equation}
\Qphotcal(\lambda_L) = \left\lbrace
\begin{array}{ll}
          \Qphotmes(\lambda_L) - \Qphot^{\mathrm{fluo}}(\lambda_L) &\ \text{if}\ \lambda_L < \SI{400}{\nano\meter} \\
         \Qphotmes(\lambda_L)(1 + \alpha(\lambda_L)) &\ \text{if}\ \lambda_L \in \left[532, 644\right]\mathrm{\,nm} \\
        \Qphotmes(\lambda_L)(1 + \beta(\lambda_L)) &\ \text{if}\ \lambda_L > \SI{1064}{\nano\meter} \\
        \Qphotmes(\lambda_L)&\ \text{elsewhere}
\end{array}\right. 
\end{equation}

All uncertainties from the evaluation of all these terms were propagated. The summary of the error budget on the CBP response is decomposed in Figure~\ref{fig:cbp_budget} as a function of laser wavelength $\lambda_L$. Systematics coming from the wavelength calibration are not represented. Indeed, as the CBP response varies slowly with wavelength, it is negligible compared to others.  In the visible range, scattered light systematics dominates. In the far infrared, the subtraction of the $\lambda_{\mathrm{comp}}$ photons is the main systematic uncertainty, while in the UV range fluorescence correction systematic dominates.

%cbp_paper_plots.ipynb
\begin{figure}[h]
    \centering
    \includegraphics[width=\columnwidth]{fig/cbp_error_budget.png}
    \caption{Total error budget for CBP response.}
    \label{fig:cbp_budget}
\end{figure}

\subsection{CBP response}

The final CBP response in output photons per Coulomb unit in the photodiode is
\begin{equation}
    \Rcbp(\lambda_c) = \frac{\Qsolarcal(\lambda_c)}{\Qphotcal(\lambda_c) \times \epsilon_{\mathrm{solar}}(\lambda_c) \times e}.
    \label{eq:rcbp2}
\end{equation} 
It can be computed for each laser burst. We averaged the values to increase the signal-to-noise ratio and get the red smooth curve presented in Figure~\ref{fig:cbp_response}:
\begin{equation}
    \overline{\Rcbp(\hat{\lambda}_c)} = \left\langle\frac{\Qsolarcal(\lambda_c)}{\Qphotcal(\lambda_c) \times \epsilon_{\mathrm{solar}}(\lambda_c) \times e}\right\rangle_{\lambda_c\in [\lambda,\lambda+\delta \lambda]}.
    \label{eq:rcbp3}
\end{equation} 
The average is performed on every burst of the three runs, with a $5\sigma$ clipping, and $\hat{\lambda}_c$ is the mean calibrated wavelength in a $\delta \lambda = \SI{1}{\nano\meter}$ bin. All uncertainties are combined in quadrature (Figure~\ref{fig:cbp_response} bottom). The total uncertainty is between 3 and $5\,$permil \todo{Check this statement at the end of the analysis as only the chromatic part is important.} in the visible range, due to scattered light systematic, and slightly higher in the UV and IR ranges.



%cbp_paper_plots.ipynb
\begin{figure}[h]
    \centering
    \includegraphics[width=\columnwidth]{fig/cbp_response.pdf}
    \caption{Top: CBP response $\overline{\Rcbp(\hat{\lambda}_c)}$ obtained with the \bpinhole and $\delta \lambda = \SI{1}{\nano\meter}$. Bottom: relative uncertainties.}
    \label{fig:cbp_response}
\end{figure}

%\input{sections/4_instrument_model}
%\clearpage

\section{Stardice response}
\label{sec:rsd}

In this part, we will show how one can obtain the \SD telescope response $\Rtel$ from the Equation~\ref{eq:rsd}. The quantities $\Qccd$ and $\Qphot$ are defined in table \ref{tab:quantities}, and $\Rcbp$ is the CBP response.

\subsection{Data set description}
\label{sec:sd_datadesc}

As described in section \ref{sec:cbp_datadesc}, the laser will still emits pulses at a rate of \SI{1}{\kilo\hertz}. The operational mode is the same, we shoot bursts of pulses that are separated with dark times. We show in Figure~\ref{fig:ccd_examples} examples of images obtained when shooting in the \SD CCD camera. 

\begin{figure}[h]
    \centering
    \includegraphics[width=\columnwidth]{fig/ccd_examples.pdf}
    \caption{Examples of images obtained when shooting in the \SD telescope with the CBP at \SI{450}{\nm}. From left to right, we have respectively an image for the \SI{5}{\mm}, \SI{2}{\mm} and \SI{75}{\micro\meter} pinhole.}
    \label{fig:ccd_examples}
    %/stardice/analysis/cbp_paper/golden_sample_analysis/dr2/npulses_stardice.ipynb
\end{figure}

The number of pulses here is also set to maintain a roughly constant cumulated charge in the photodiode at every wavelength. We had to deal with the saturation of the CCD camera and we show in the Figure~\ref{fig:saturation_lim} the maximal value in the CCD camera for one pulse. The CCD camera being highly sensitive, two pulses can be enough to saturated the pixels enlightened at some wavelengths. Our capacity to have a constant cumulated charge in the CCD camera is then limited by the finite value of pulses that we can send. The pinhole diameter, the point of impact on the \SD primary mirror, the point of incidence on the \SD focal plane and the bands studied can vary from a measurement to another. All these informations can be found in table \ref{tab:schedule}.

\begin{figure}[h]
    \centering
    \includegraphics[width=\columnwidth]{fig/saturation_lim_ccd.pdf}
    \caption{Maximal pixel value in ADU measured in the CCD camera of \SD normalized by the number of pulses and the saturation limit of the camera estimated at \SI{61600}{ADU}, against the wavelength. Here we can see that for some wavelengths above \SI{668}{\nano\meter} a few number of pulses can saturate the camera.}
    \label{fig:saturation_lim}
    %/stardice/analysis/cbp_paper/golden_sample_analysis/dr2/npulses_stardice.ipynb
\end{figure}


\begin{itemize}
\item Images
\item Photocurrent timeseries
\item Spectral time series
\end{itemize}

\subsection{Reduction of images}
\label{sec:photometry}

In this section, the data reduction for the monitoring photodiode and the spectrograph are the same as described respectively in the section \ref{sec:pd_reduction} and \ref{sec:spectro_reduction}. Aperture photometry is done to extract the cumulated charge from the \SD camera images. This requires to estimate the background and subtract it and the background subtraction method depends on the pinhole used at the input of the CBP optics. We will described the two methods in the sections below.

\subsubsection{\spinhole pinhole}

Firstly, we load a batch of images from the same run. We compute the mean of the pixels in the overscan and subtract it to our images. We estimate $\sigma$ the standard deviation of an image, and every pixel with a signal higher than 5$\sigma$ is masked, as well as all the surrounding pixels that has a signal higher than 2$\sigma$. Then we proceed to a segmentation of the masked image into boxes of $256\times256$ pixels. We estimate the mean and the standard deviation of the background in each of these boxes, and we interpolate their values in a 2D map to get our estimation of the background. This background is subtracted to the image, then we compute the centroid of the spot of interest. We measure the number of charges $\Qccd$ with aperture photometry at a radius of 21.2 pixels (see growth curve in Figure~\ref{fig:growth_75um}).

\begin{figure}[h]
    \centering
    \includegraphics[width=\columnwidth]{fig/growth_curve_75um.pdf}
    \caption{Growth curve obtained with the \spinhole pinhole. Le vertical axis is obtained by computing 1 minus the flux in ADU at the radius aperture in the horizontal axis $F$, normalized by the flux maximal at the maximal aperture $F_{MAX}$. Here the maximal aperture corresponds to a radius of 56.4 pixels. The scale is linear under 0.05, and logarithmic above.}
    \label{fig:growth_75um}
    %/stardice/analysis/cbp_paper/golden_sample_analysis/dr2/growth_curves.ipynb
\end{figure}

\subsubsection{\bpinhole pinhole}

When using \bpinhole pinhole, the main spot take a significant area of the focal plane, meaning there is not enough background in the image to estimate it. This time we use images without any light coming from the CBP to estimate the background. Between each image taken with light coming from the CBP, we took background images by turning off the laser. These background images were stationary with respect to time and wavelength, as it can be seen respectively in Figure~\ref{fig:stationarity_time} and \ref{fig:stationarity_wl}. We compute a master background by doing the mean from all of these background images. Then we find the centroid of the spot of interest, and we measure the charges collected in the \SD camera by aperture photometry. We evaluate the optimal radius at 300 pixels for the \bpinhole pinhole in order to contain the main spot and the ghost (see growth curve \ref{fig:growth_5mm}). We estimate the background by doing the aperture photometry at the same position and same radius of the master background, and then subtract it to the charge measured. Finally, we can extract the charges $\Qccd$ collected in the \SD camera coming from the CBP. The same method is applied for the \SI{2}{\milli\meter} pinhole, with an aperture of 150 pixels.

\begin{figure}[h]
    \centering
    \includegraphics[width=\columnwidth]{fig/growth_curve_5mm.pdf}
    \caption{Growth curve obtained with the \spinhole pinhole. Le vertical axis is obtained by computing 1 minus the flux in ADU at the radius aperture in the horizontal axis $F$, normalized by the flux maximal at the maximal aperture $F_{MAX}$. Here the maximal aperture corresponds to a radius of 490 pixels. The scale is linear under 0.01, and logarithmic above.}
    \label{fig:growth_5mm}    
    %/stardice/analysis/cbp_paper/golden_sample_analysis/dr2/growth_curves.ipynb
\end{figure}

\com{Courbe de croissance -> problème dans l'IR : autres représentations ? ; 300 pixels contient le ghost}

\begin{figure}[h]
    \centering
    \includegraphics[width=\columnwidth]{fig/background_stationary_time.pdf}
    \caption{Charge per pixel for each background image in \SD with overscan correction with respect to the time.}
    \label{fig:stationarity_time}
    %/stardice/analysis/cbp_paper/golden_sample_analysis/dr2/background_and_centroid_stationarity.ipynb
\end{figure}

\begin{figure}[h]
    \centering
    \includegraphics[width=\columnwidth]{fig/background_stationary_wavelength.pdf}
    \caption{Charge per pixel for each background image in \SD with overscan correction with respect to the wavelength.}
    \label{fig:stationarity_wl}
\end{figure}

\subsection{Correction of the \SI{532}{\nano\meter} contamination}

When extracting the charges $\Qccd$, we have to deal with the laser contamination explained in the section \ref{sec:532_cont}. The goal is to build a model for the \SI{532}{\nano\meter} contribution observed in the range [532 - 644] nm. As the photodiode and the solar cell, the total charge measured in the \SD camera $\Qccd$ is the sum of the charges from the main wavelength line $\Qccd^{\lambda_L}$ and the charges from the \SI{532}{\nm} contamination $\Qccd^{532}$. This statement cooresponds to the Equation~\ref{eq:qccd_mes}. Again, the charge measured in the photodiode $\Qphot$ is defined by the Equation~\ref{eq:qphot_mes}.
 
\begin{equation}
    \Qccd = \Qccd^{\lambda_L} + \Qccd^{532}
    \label{eq:qccd_mes}
\end{equation}

As we defined the final calibrated amount of charges detected in the photodiode per laser burst $\Qphot^{\rm cal}$, we want to do the same with the \SD CCD. The light detected by the \SD CCD has gone through the CBP optics and the \SD telescope. These instruments have respectively an optical response defined as $\Rcbp(\lambda)$ and $\Rtel(\lambda)$. We can define $R(\lambda)$ the total response of the CBP and the \SD telescope optics as in Equation~\ref{eq:rtot}. 

\begin{equation}
    R(\lambda) = \Rtel(\lambda) \times \Rcbp(\lambda)
    \label{eq:rtot}
\end{equation}

The charges $\Qccd$ can be expressed as a function of $\Rcbp$, $\Rtel$ and $\Qphot$, as shown in Equation~\ref{eq:qccd}. 

\begin{equation}
    \Qccd(\lambda) = R(\lambda) \times \Qphot(\lambda)
    \label{eq:qccd}
\end{equation}

We can compute the response $R(\lambda=532)$ can be computed thanks to the Equation~\ref{eq:rsd} because $\lambda_L = \lambda_\mathrm{comp} = 532$ nm, so all the charges detected are at the main wavelength. By combining Equations~\ref{eq:qccd_mes}, \ref{eq:qccd} and $R(\lambda=532)$, and using the $\alpha(\lambda)$ factor from \ref{eq:alpha}, we can obtain the following equation.

\begin{equation}
\begin{aligned}
    \Qccd^{\lambda_L} & = \Qccd - R(\lambda=532) \times \Qphot^{532} \\ 
    & = \Qccd - R(\lambda=532) \times \alpha(\lambda) \times \Qphot^{\lambda_L} \equiv \Qccd^{\rm cal}
    \label{eq:qccd_final}
\end{aligned}
\end{equation}

With this Equation~\ref{eq:qccd_final} we have obtained an expression for the final calibrated amount of charges detected in the \SD CCD $\Qccd^{\rm cal}$. The same can be done with the complementary wavelength line $\lambda_{\rm comp}$ appearing after $\lambda_L > \SI{1064}{\nm}$, where we use the correction coefficient $\beta$ from Equation~\ref{eq:beta} instead of $\alpha$.

\begin{figure}[h]
    \centering
    \includegraphics[width=\columnwidth]{fig/g_filter_532.pdf}
    \caption{StarDice g filter transmission as a function of the set laser wavelength $\lambda_L$, with and without the \SI{532}{\nm} correction. For clarity, we added a zoom on the out-of-band transmission where $\SI{532}{\nm}$ line is transmitted while higher wavelengths are shot by the laser.}
    \label{fig:g_filter_532}
    %cbp_paper_plots.ipynb
\end{figure}

\subsection{Intercalibration of the 75 µm and 5 mm pinholes}

In Section~\ref{sec:strategy} we said that there is a need to change the pinhole in the CBP slides according to the measurement we want to do, as refered in Table~\ref{tab:schedule}. We need to ensure that we understand how the CBP response has changed when we switched the pinhole. We took measurements with the \SD telescope with both \bpinhole pinhole and \spinhole pinhole in order to intercalibrate the two pinholes, and evaluate the systematic related to this change of pinhole diameter. One major difference between the \spinhole pinhole and the \bpinhole pinhole is how the ghost and the main spot will superimpose on the focal plane.  We show in Figure~\ref{fig:schema_ghost} the optical path that produces the 1\up{st} order ghost on the focal plane. In Figure~\ref{fig:ghost_contrast}, we can see an image of the 1\up{st} order ghost next to the main spot obtained with the \spinhole pinhole. The distance between the centroïd of the main spot and the centroïd of the 1\up{st} order ghost is between 30 to 70 pixels depending on the radial position on the mirror. In the mean time, the \bpinhole pinhole photometry is made with an aperture of 300 pixels, containing the 1\up{st} order ghost and the main spot. 

\subsubsection{First order ghost photometry}

 We want to do the photometry of the 1\up{st} order ghost with the \spinhole pinhole to estimate the contribution of the ghost when we do the photometry for the \bpinhole pinhole. We build a mask with the expected ghost shape, and we fit its best position on the image. The signal measure from the 1\up{st} order ghost is simply the sum of the ADUs contained in this mask, after background subtraction. To estimate and subtract the background, we make the hypothesis that the main spot has a vertical symmetry, so we expect the background to be the same at the right and the left side of a vertical line fixed at the main spot coordinates. We place a second mask at the symmetric position from this axis of symmetry, and we measure the charges at this position, which correspond to background only. This background measurement is then subtract from the ghost photometry. The main spot emasurement is made by aperture photometry as explained in Section~\ref{sec:photometry}. 
 
 Let be $G_0$ the quantity of light collected in the main spot, and $G_\mathrm{n}$ the quantity of light collected in the ghost of order $n$. We define $\Kghost(\lambda)$ the ratio of the 1\up{st} order ghost $G_1$ over the main spot $G_0$ shown in Equation~\ref{eq:ratio_ghost}. We show this ratio measured for 5 different measurements in Figure~\ref{fig:ghost_ratio}.

\begin{equation}
    \Kghost(\lambda) = \frac{G_1(\lambda)}{G_0(\lambda)}
    \label{eq:ratio_ghost}
\end{equation}

\begin{figure}[h]
    \centering
    \includegraphics[width=\columnwidth]{fig/schema_ghost.pdf}
    \caption{Schematic of the origin of the ghost. A portion of the incident light reflects on the surface of the CCD, and then reflects again on one of the two faces of the window, and are then absorbed by the CCD. This creates a defocused and less intense image at an other position in the focal plane. Everytime light is absorbed by the CCD, a portion is reflected and generate a new ghost.}
    \label{fig:schema_ghost}
    %~/stardice/analysis/cbp_paper/golden_sample_analysis/dr2/ghost_photometry_general.ipynb
\end{figure}

\begin{figure}[h]
    \centering
    \includegraphics[width=\columnwidth]{fig/ghost_contrast.pdf}
    \caption{Image of the main spot in the center and the first order ghost at its left. The level max of the color bar is set low intentionally to make the ghost visible. The distance between the first order ghost and the main spot is between 30 and 70 pixels depending on the radial position of the point of impact on the \SD primary mirror.}
    \label{fig:ghost_contrast}
    %~/stardice/analysis/cbp_paper/golden_sample_analysis/dr2/ghost_photometry_general.ipynb
\end{figure}


 \begin{figure}[h]
     \centering
     \resizebox{\hsize}{!}{\includegraphics{fig/ghost_ratio.pdf}}
     \caption{Ratio $\Kghost$ (which correspond to the order 1 ghost $G_{1}$ over the main spot $G_0$) with respect to wavelength in nanometer. The mean spline goes through five datasets, two at different radial positions on the mirror and three at the same position. All responses are at the same position on the focal plane.}
     \label{fig:ghost_ratio}
    %~/stardice/analysis/cbp_paper/golden_sample_analysis/dr2/ghost_photometry_general.ipynb
 \end{figure}
 
 \subsubsection{Contribution of the n\up{th} order ghosts}

 Now that we did the photometry for the first order ghost, we want to study the higher orders. Since the intensity of these ghosts is very weak and hard to analyse, we will simulate their contribution from the $1^{\mathrm{st}}$ order ghost analysis. Let be $\Rwindow$ the reflection coefficient at the interface air-window and window-air, and $\Rccd$ the reflection coefficient at the interface air-CCD. We know the transmission of the window $T_\mathrm{window}$ from manufacturer datasheets, so we can infer $\Rwindow$ with the Equation~\ref{eq:rwindow}.
 
\begin{equation}
    \Rwindow = 1 - T_\mathrm{window}
    \label{eq:rwindow}
\end{equation}

 
 We define $\phi$ the flux collected by the camera, $G_0$ the quantity of light collected in the main spot, and $G_\mathrm{n}$ the quantity of light collected in the ghost of order $n$. All these quantities are linked according to the following Equations~\ref{eq:g0} and \ref{eq:gn}. The sum of all the ghosts $G_{\mathrm{n>0}}$ is defined in the Equation~\ref{eq:sum_ghost}
 \footnote{If |q| < 1, the serie $\left( \sum_{n=0}^{m} q^n \right)_{\mathrm{m \in \mathbb{N}}}$ strictly converge and \\ $\sum_{n=0}^{\infty} q^n \equiv \lim\limits_{m \rightarrow \infty} \sum_{n=0}^{m} q^n = \frac{1}{1-q}$}. %, with $R_{\mathrm{ghost}} = [2 \Rccd \Rwindow - \Rccd \Rwindow^{2}]$.

 \begin{equation}
     G_0 = (1-\Rwindow)^{2} \times (1-\Rccd) \times \phi
     \label{eq:g0}
 \end{equation}

\begin{equation}
\begin{aligned}
    G_\mathrm{n} & = G_{0} \times [\Rccd \Rwindow + \Rccd(1-\Rwindow)\Rwindow]^{n} \\
    & = G_{0} \times [2 \Rccd \Rwindow - \Rccd \Rwindow^{2}] ^{n} \\
     \label{eq:gn}
\end{aligned}
\end{equation}

 \begin{equation}
 \begin{aligned}
     G_{n>0} & = \sum_{n=0}^{\mathrm{n} \rightarrow \infty} G_\mathrm{n} - G_0 \\
     & = G_0 \times \sum_{n=0}^{n \rightarrow \infty} \left( [2 \Rccd \Rwindow - \Rccd \Rwindow^{2}] ^{n} - G_0 \right)\\
     & = G_0 \times\left( \frac{1}{1- [2 \Rccd \Rwindow - \Rccd \Rwindow^{2}]} - 1\right)
     \label{eq:sum_ghost}
 \end{aligned}
 \end{equation}

 \begin{equation}
     G_{n>1} = G_{n>0} - G_1
     \label{eq:sum_ghost_sup_1}
 \end{equation}

As we do the photometry of the 1\up{st} order ghost $G_1$, we want to verify that the ghosts of higher orders $G_{\mathrm{n}>1}$ are negligible. Using the Equations~\ref{eq:ratio_ghost}, \ref{eq:g0} and \ref{eq:gn}, we can compute $\Rccd$ with Equation~\ref{eq:rccd}.

\begin{equation}
    \Rccd = \frac{\Kghost}{\Rwindow (2 - \Rwindow)}
    \label{eq:rccd}
\end{equation}

We measured $G_1$ with aperture photometry, and we combine equations \ref{eq:rwindow}, \ref{eq:sum_ghost}, \ref{eq:sum_ghost_sup_1}, and \ref{eq:rccd} in order to compute $G_{\mathrm{n}>1}$. We show the ratio $K_{G_{\mathrm{n}>1}/G_0}$ in Figure~\ref{fig:ratio_ginf_g0}. We see that $K_{G_{\mathrm{n}>1}/G_0}$ is always below the per mil level, so we can neglect the contribution of the ghosts at order n>1. We do the hypothesis that the behaviour of the ghost is the same for the \SI{2}{\milli\meter} pinhole.

\begin{figure}[h]
    \centering
    \includegraphics[width=\columnwidth]{fig/ratio_g1_ginf.pdf}
    \caption{Ratio $K_{G_{\mathrm{n}>1}/G_0}$ (which correspond to the sum of the ghosts at order higher than 1 $G_{\mathrm{n}>1}$ over the main spot $G_0$) with respect to the wavelength in nanometer.}
    \label{fig:ratio_ginf_g0}
    %~/stardice/analysis/cbp_paper/golden_sample_analysis/dr2/ghost_convergence.ipynb   
\end{figure}

\subsubsection{Ghost contribution for \bpinhole pinhole}

The total charges $\Qccd{}_{, \, \SI{5}{\milli\meter}}$ measured by doing the aperture photometry at 300 pixels for the \bpinhole pinhole is the sum of the charges $Q_0$ in the main spot $G_0$ and the charges $\Qghost$ in the 1\up{st} order ghost $G_1$. These quantities are defined by the following equations \ref{eq:qghost} and \ref{eq:qmain}, with $\Eccd$ the quantum efficiency of the StarDice CCD camera. We use the Equation~\ref{eq:ratio_ghost} to develop the Equation~\ref{eq:qccd_5mm}. Thanks to Equation~\ref{eq:qccd_5mm} we can extract $Q_0$ and study only the charges from the main spot.

\begin{equation}
    \Qghost = G_1 \times \Eccd
    \label{eq:qghost}
\end{equation}

\begin{equation}
    Q_0 = G_0 \times \Eccd
    \label{eq:qmain}
\end{equation}

\begin{equation}
\begin{aligned}
    Q_0 & = \Qccd{}_{, \, \SI{5}{\milli\meter}} - \Qghost \\
    & = \frac{\Qccd{}_{, \, \SI{5}{\milli\meter}}}{1 + \Kghost}
    \label{eq:qccd_5mm}
\end{aligned}
\end{equation}

Once we have corrected the \bpinhole pinhole from the ghost contribution, we can compute the ratio between the \SD responses defined as $\Kpinholes(\lambda)$ in Equation~\ref{eq:ratio_pinholes}. Thanks to this equation, we can infer $\Rcbp^{\spinhole}(\lambda)$ from the measurement of $\Rcbp^{\bpinhole}(\lambda)$.

\begin{equation}
    \Kpinholes(\lambda) = \frac{\Rcbp^{\bpinhole}(\lambda)}{\Rcbp^{\spinhole}(\lambda)}
    \label{eq:ratio_pinholes}
\end{equation}

We show these ratios before and after the ghost correction in the upper Figure~\ref{fig:ratio_pinholes}. We expect the ratio to be linear since it should be the ratio of two surfaces, so we draw a linear spline through the data. In the ultraviolet below \SI{400}{\nm} where the window is highly reflective, we see that the ghost correction has flattened the ratios. Above \SI{900}{\nm} we can see a significative difference between the ratio and the linear spline. This phenomenon is not quite understood and is discussed in section \ref{sec:discussion}, but we precise that the linear fit is made only for the wavelengths below \SI{900}{\nano\meter}. 

\begin{figure}[h]
    \centering
    \includegraphics[width=\columnwidth]{fig/ratio_pinholes.pdf}
    \caption{Ratio $\Kpinholes(\lambda)$ with respect to wavelength, before and after ghost correction. We compute a linear spline that best fit the data between \SI{400}{\nm} and \SI{900}{\nm}.}
    \label{fig:ratio_pinholes}
    %~/stardice/analysis/cbp_paper/golden_sample_analysis/dr2/ratio_pinholes.ipynb
\end{figure}

\subsubsection{Summary}

The summary of the error budget on the \SD telescope response id detailed in Figure~\ref{fig:sd_budget}


\begin{figure}
    \centering
    \includegraphics[width=\columnwidth]{fig/sd_uncertainties_budget.pdf}
    \caption{Total error budget for \SD response.}
    \label{fig:sd_budget}
\end{figure}

\subsection{\SD response scan}

\subsubsection{\SD optics, filters and grating responses}

In this section we will present the results of our different measurements. The following measurements are all taken at the same position on the mirror and the focal plane. The Figure~\ref{fig:stardice_5mm_response} show the \SD response obtained with the \bpinhole pinhole, and no filter. The Figure~\ref{fig:stardice_75um_response} show the \SD response obtained with the \spinhole pinhole, with all filters. In this figure we have a wavelength resolution sufficient enough to see the filter edges. The Figure~\ref{fig:stardice_grating_response} show the \SD response obtained with the \spinhole pinhole, with the grating set in front of the camera. The grating being blazed so that the 1\up{st} order of diffraction is the brightest, so we will mainly focus on this order of diffraction. We see a cut of the 2\up{nd} and 3\up{rd} order that correspond to the wavelength at which the signal is outside the CCD sensor.

\begin{figure}[h]
    \centering
    \includegraphics[width=\columnwidth]{fig/stardice_5mm_response.pdf}
    \caption{Up : \SD response with no filter and \bpinhole pinhole with respect to wavelength in nanometer. Bottom : Uncertainties over the \SD response measurement with respect to wavelength in nanometer.}
    \label{fig:stardice_5mm_response}
    %~/stardice/analysis/cbp_paper/golden_sample_analysis/dr2/response_plots.ipynb
\end{figure}

\begin{figure}[h]
    \centering
    \includegraphics[width=\columnwidth]{fig/stardice_75um_response.pdf}
    \caption{Up : \SD response with at all filters and \spinhole pinhole with respect to wavelength in nanometer. Bottom : Uncertainties over the \SD response measurement with respect to wavelength in nanometer.}
    \label{fig:stardice_75um_response}
    %~/stardice/analysis/cbp_paper/golden_sample_analysis/dr2/response_plots.ipynb
\end{figure}

\begin{figure}[h]
    \centering
    \includegraphics[width=\columnwidth]{fig/stardice_grating_response.pdf}
    \caption{Up : \SD response with the grating in front of the camera and \spinhole pinhole with respect to wavelength in nanometer. Bottom : Uncertainties over the \SD response measurement with respect to wavelength in nanometer.}
    \label{fig:stardice_grating_response}
    %~/stardice/analysis/cbp_paper/golden_sample_analysis/dr2/response_plots.ipynb
\end{figure}

\subsubsection{Radial positions}

The upper Figure~\ref{fig:radial_positions} show the \SD response obtained with the \spinhole pinhole and without filters for the different radial positions on the mirror shown in Figure~\ref{fig:8_mirror_positions} left, and the same focal plane position. The lower figure show the normalized residuals to the mean spline going through the four responses. In the ultraviolet below \SI{450}{\nm}, we can see a significant difference between the four radial positions. It goes up to 30\%, and it will be discussed in the section \ref{sec:discussion}. What we see above \SI{1000}{\nm} in the infrared is the result of the data oscillations around the spline caused by the fringing. 

\begin{figure}[h]
    \centering
    \includegraphics[width=\columnwidth]{fig/radial_positions.pdf}
    \caption{Top : \SD response for the different radial positions on the mirror, but same position on the focal plane. Bottom : Relative difference between the data and the mean spline.}
    \label{fig:radial_positions}
    %~/stardice/analysis/cbp_paper/golden_sample_analysis/dr2/2022_03_01_stardice_transmission_radius.ipynb
\end{figure}

\subsubsection{Quadrant positions}

The upper Figure~\ref{fig:radial_positions} show the \SD response obtained with the \spinhole pinhole and without filters for the different quadrant positions on the mirror shown in Figure~\ref{fig:8_mirror_positions} right, and the same focal plane position. The lower figure show the normalized residuals to the mean spline going through the four responses. We see again a divergence below \SI{400}{\nm} of about 20\% at most. The phenomenon in the infrared is still caused by the fringing. 

\begin{figure}[h]
    \centering
    \includegraphics[width=\columnwidth]{fig/quadrant_positions.pdf}
    \caption{Top : \SD response for the different quadrant positions on the mirror, but same position on the focal plane. Bottom : Relative difference between the data and the mean spline.}
    \label{fig:quadrant_positions}
    %~/stardice/analysis/cbp_paper/golden_sample_analysis/dr2/2022_03_01_stardice_transmission_mirror_samples.ipynb
\end{figure}

\subsubsection{Focal plane positions}

The upper Figure~\ref{fig:ccd_positions} show the \SD response obtained with the \spinhole pinhole and without filters for the same radial and quadrant position on the mirror, and the different focal planes positions from the grid show in Figure~\ref{fig:ccd_grid}. The lower figure show the normalized residuals to the mean spline going through the sixteen responses.

\begin{figure}[h]
    \centering
    \includegraphics[width=\columnwidth]{fig/ccd_positions.pdf}
    \caption{Top : \SD response for the different positions on the focal plane, but same position on the mirror. Bottom : Relative difference between the data and the mean spline.}
    \label{fig:ccd_positions}
    %~/stardice/analysis/cbp_paper/golden_sample_analysis/dr2/2022_03_09_stardice_transmission_grid_auto.ipynb
\end{figure}

\subsection{Systematic uncertainties}
\label{sec:systematics}

\subsubsection{Stability of the StarDice responses}

\subsubsection{Gain and linearity}
\label{sec:gain}

Varying pinhole with StarDice, and CBP
Varying QSW but depending on result it falls into this subsection or the following

\subsubsection{Pinhole chromaticity}

\subsubsection{Pull distributions}

\subsubsection{Courbes de croissances}






\section{Synthesis of the equivalent transmission for full pupil illumination}
\label{sec:pupil_stitching}

Both the telescope reflectivity and filter transmission display a very
clear radial dependency. While the latter is expected due to the
filters' interferometric nature, the former's origin is not
yet understood. We can nevertheless build an empirical model assuming
smooth transitions between the measurements and compute a
theoretical "full pupil" transmission from there by averaging the model over the
illuminated portion of the primary mirror. As is demonstrated in Table
TX, these two steps are necessary to reach sub-percent colour accuracy and sub-nm accuracy in central wavelengths. Last, we proceed
with propagating the statistical and systematic errors on the final
transmission curves. The impact of these errors in the interpretation
of broadband photometry is better understood when integrated as errors
on the filter normalisation and central wavelength, which can readily
be translated into errors in colors and color-terms.


\subsection{Radial model of the instrument transmission}
\label{sec:model}

The open transmission of the telescope is modelled as a smooth function
of wavelength and incidence angle. The function is developed on a 2D
cubic B-spline basis, with \num{35} wavelength nodes regularly spaced
between \SIrange{350}{1100}{nm}, and 2 nodes in angles between
\SI{1.97}{\degree} and \SI{7.24}{\degree}. The two angles correspond
to the inner and outer edge of the occultation-free primary mirror,
set at \SIrange{55}{203}{mm} in radius assuming a \SI{1600}{mm} focal
length.

For data acquired with a filter in the path, the open transmission
model is multiplied by a model of the interference filter transmission
build as follows:
\begin{equation}
  \label{eq:filtertransmission}
T(\lambda, \theta) = \mathcal T\left(\frac{\lambda}{\sqrt{1 -
    (\sin(\theta) / n_\text{eff})^2}}\right)
\end{equation}
where $n_\text{eff}$ is an effective index for the inserted filter,
and $\mathcal T$ is a piece-wise linear function of wavelength. The piece-wise linear function is first developed on a regular grid with
\num{150} nodes between \SIrange{350}{1100}{nm}, giving a general
resolution of \SI{5}{nm} and is then further refined where needed by
equally splitting in two intervals where the local mean chi-square
exceeds the global mean chi-square by more than $3$ standard
deviations. This refinement process is repeated \num{4} times so that
filter fronts typically end-up being modelled with up to \SI{\sim
  0.3}{nm} resolution.

Last, photometry of the grating zeroth and first orders are adjusted
with the open transmission multiplied by a cubic B-spline decomposed
on 105 nodes in wavelength.

The composite model is fit to dataset number 2, encompassing data at 4
different radii and successive observations without filters and with
all 7 filters and grating. The baseline model has \num{1979} free
parameters, \num{148} to describe the open transmission, \num{420} to
describe each of the grating order, and about $6\times165$ to describe
each of the $ugrizy$ filters. The $u$ band filters blue edge cannot be
reliably determined by the data. Fit results are shown in
Fig.~\ref{fig:lambdathetafitresults}.

\begin{figure*}
  \centering
  \includegraphics[width=1\linewidth]{./fig/lambdathetafitresults.pdf}
  \caption{Model of the wavelength and radial dependency of the
    StarDICE response to CBP illumination $R(\lambda)$ (see
    Eq.~\ref{eq:rtot}). Each panel display the raw measurements at
    the 4 sampled position for each of the filter configurations: no
    filters (Open), with one of the 6 photometric filters ($ugrizy$)
    or with the grating looking either at the zeroth order or the
    first order spots. The panel immediately below each panel
    display the residuals to the model. All panels are normalized to
    the peak of the response, reached in the open configuration at
    \SI{398}{nm} and \SI{6.2}{\degree}.  }
  \label{fig:lambdathetafitresults}
\end{figure*}


\begin{figure}
  \centering
  \includegraphics[width=1\linewidth]{fig/diffraction_dust.png}
  \caption{Diffraction rings due to the presence of a dust spot on the r filter. Red circles are over-plotted to ease the visualisation.}
  \label{fig:dust}
\end{figure}

The model satisfactorily describes the dataset. The most significant
discrepancy is a noticeable gray decrease of the transmission of the
$r$ filters for the sample measurement at \SI{3.7}{\degree} with
respect to the average of the other three. Investigating this issue,
we found that the PSF tails in the corresponding images display a
diffraction figure consistent with the beam being intercepted by a
dust particle on the filter surface with a diameter in the range
\SIrange{200}{300}{\micro\meter} (Figure~\ref{fig:dust}). The approximate area of the beam
spot being \SI{12}{mm\squared}, the estimated particle sizes predicts
a decrease in transmission of up to \SI{0.6}{\percent}, very
consistent with the observed decrease. Very similar decrease are
observed for some observations in $u,g$ and $y$. Although in these
cases it was not possible to estimate the particle size from
diffraction features in the images, we think that similar dust
contamination is at play. The size of the ``gray'' discrepancies are
easier to read on the top panel of Fig.~\ref{fig:metrics}, which
displays the difference in the transmission integral between the
measurement and the model at each position. The standard deviation of
the model/measurement discrepancies is \SI{10.2}{mmag}. Taking this
number as an estimate of dust-induced dispersion, we can deduce that
the corresponding uncertainty on the per-filter normalisation of the
model, which average 4 independent samples, is of the order of
\SI{5}{mmag}.

The bottom panel in Fig.~\ref{fig:metrics} displays the difference in
central wavelength between the model and the measurement for all 4
radius in the central panel of Fig.~\ref{fig:metrics}. The model
successfully reproduces the central wavelength of all filters with a
standard deviation of \SI{0.13}{nm} which delivers an improvement by
an order of magnitude with respect to a model neglecting the
wavelength shift. 

\begin{figure}
  \centering
  \includegraphics[width=1\linewidth]{fig/metrics.pdf}
  \caption{Discrepancies between model and raw measurements at different mirror locations summarized according to 3 metrics: summarize}
  \label{fig:metrics}
\end{figure}

\subsection{Full pupil synthetic transmission curves}

The full pupil transmission is synthesized by numerically averaging
the above model assuming that the pupil is a perfect annulus with an
inner radius of \SI{55}{mm} and an outer radius of \SI{203}{mm}. A
rectangular quadrature with 100 evenly sampled points in radius has
been used for the averaging. The curves have been normalized using the
CBP response from Sect. (corrected for the \SI{5}{mm} to
\SI{75}{\micro\meter} pinhole response change determined in Sect. TX.,
and the aperture fraction estimated in Sect.Tx).  An estimate of the
absolute transmission curve is then computed assuming an effective
mirror area of TX. The resulting transmission curves are shown in
Fig.~\ref{fig:fullpupiltrans}.
\begin{figure}
  \centering
  \includegraphics[width=1\linewidth]{fig/fullpupill.pdf}
  \caption{Full-pupill transmission curves for the StarDICE instruments.}
  \label{fig:fullpupiltrans}
\end{figure}

\todo{
We also tested the model against the 4 remaining mirror samples that
are not part of the training dataset. The result of the test is
summarized in Fig.\ref{fig:metrics}.}


\subsection{Final error budget}


\section{Discussion}
\label{sec:discussion}

\subsection{Modelisation of the \SD data for \spinhole pinhole}

Based on the work from \cite{moffat}, the \SD telescope PSF can be modelized as a moffat distribution. When shooting with the \spinhole pinhole inside the \SD telescope, the flux $F(r, \lambda)$ measured in the CDD with aperture photometry at a radius $r$ and a wavelength $\lambda$, can be modelized as: 

\begin{equation}
F(r, \lambda) = \frac{A(\lambda) \times M(r, \lambda) + \Kghostfit(r, \lambda)}{1 + \Kghostfit(r \rightarrow +\infty, \lambda)} + \bkg(r, \lambda),
\label{eq:moffat_model}
\end{equation}
with $A(\lambda)$ the total amplitude, $\Kghostfit(r, \lambda)$ the relative contribution of the 1\up{st} order ghost to $A(\lambda)$, $\bkg(r, \lambda)$ the contribution of the background and $M(r, \lambda)$ the moffat distribution defined as:

\begin{equation}
M(r, \lambda)= \left( 1+\frac{r^2}{\alpha(\lambda)^2} \right)^{-\beta(\lambda)},
\end{equation}
with $\alpha(\lambda)$ and $\beta(\lambda)$ the parameters of the Moffat distribution. The flux is normalized by $(1 + \Kghostfit(r \rightarrow +\infty, \lambda))$ so its infinite integral is equal to the maximum amplitude $A(\lambda)$. 

The core of the PSF is actually more complex than a moffat distribution because of the intrisinc shape of the CBP output. However the tail of the PSF should not be impacted by this, and is expected to behave like a Moffat distribution. 
The model is fitted on aperture photometry for a radius $r=\SI{20.9}{pixels}$, and then annulus of external radius from \SI{24.9}{pixels} to \SI{419.1}{pixels}. These radius are regularly spaced on a logarithm scale, as shown in Figure~\ref{fig:apertures_fit}. To take in account $\Kghostfit(r, \lambda)$, an additive value is set free for every annulus that contains a ghost contribution. The parameters $A(\lambda)$, $\alpha(\lambda)$, $\beta(\lambda)$, $\Kghostfit(\lambda)$ and $\bkg(\lambda)$ are adjusted with the \spinhole pinhole dataset No.~8 from Table~\ref{tab:schedule}. 

\begin{figure}[h]
     \centering
     \resizebox{\hsize}{!}{\includegraphics{fig/apertures_fit.pdf}}
     \caption{Image of the \spinhole observed with the \SD camera at \SI{400}{\nano\meter}. The circles corresponds to the radius used to adjust the parameters of Equation~\ref{eq:moffat_model}.}
     \label{fig:apertures_fit}
    %~/stardice/analysis/cbp_paper/total_fluxes/jax_fit.ipynb
\end{figure}

The evolution of the Moffat distribution is expected to be smooth with respect to wavelength. So the parameters $\alpha(\lambda)$ and $\beta(\lambda)$ are developed on a B-spline basis with 20 wavelength nodes regularly spaced between \SI{350}{\nano\meter} to \SI{1100}{\nano\meter}. The same goes for the parameters $G_1(\lambda)$ as it corresponds to light coming from reflections on surfaces in the optical path. The results of this fit is shown Figure~\ref{fig:result_params}. The bottom panel corresponds to the aperture correction that is needed to apply to the aperture photometry measurement. It is around 2\% below \SI{950}{\nano\meter}, and increase by one order of magnitude in the infrared. This increase is correlated with the variation of the Moffat parameters $\alpha$ and $\beta$, showing that the PSF is drastically changing in the near infrared. This issue is discussed in more details section \ref{sec:ir}. The third panel show a constant background against wavelength, at a mean of \SI{0.27}{ADU/pixel}.

\begin{figure}[h]
     \centering
     \resizebox{\hsize}{!}{\includegraphics{fig/result_params.pdf}}
     \caption{Result of the model of Equation~\ref{eq:moffat_model} adjusted on the \spinhole pinhole dataset No.~8 from Table~\ref{tab:schedule}. The first and second panel represent respectively the $\alpha$ and $\beta$ parameters from the Moffat distribution. The third panel represent the background contribution per pixel. The fourth panel represent $\Kghostfit$, and $\Kghost$ for which the measurement has been detailed in section \ref{sec:ghost}. The fifth panel represent the missing fraction when measuring the \spinhole with aperture photometry compared to the total amplitude fitted $A$.}
     \label{fig:result_params}
    %~/stardice/analysis/cbp_paper/total_fluxes/jax_fit.ipynb
\end{figure}






\section{Conclusion}
\label{sec:ccl}


\begin{acknowledgements}
We are grateful to ...
\end{acknowledgements}

\bibliographystyle{aa}
\bibliography{cbp}

\appendix

\end{document}

%& -shell-escape
%\documentclass[a4paper]{aa}
\documentclass[onecolumn]{aa}
% \documentclass{article}
\usepackage[pagebackref,colorlinks,citecolor=blue,urlcolor=blue,linkcolor=blue]{hyperref}
\usepackage[varg]{txfonts}
\usepackage{graphicx}
\usepackage{natbib}
\usepackage{amssymb}
\usepackage{amsmath}
\usepackage{tikz}
\usepackage{subcaption}
\usepackage{lipsum}
\usetikzlibrary{arrows,positioning,shapes} %,optics}

%\usepackage[squaren,Gray]{SIunits}
\usepackage{siunitx}
%\usepackage{units} % nicefrac
%\newcommand{\units}[1]{$\mathrm{#1}$}

\newcommand{\Oldthorlabs}{SM05PD1B}
\newcommand{\Newthorlabs}{SM05PD3A}
\newcommand{\QSD}{Q_{\mathrm{SD}}}
\newcommand{\QPD}{Q_{\mathrm{PD}}}
\newcommand{\QSC}{Q_{\mathrm{SC}}}
\newcommand{\RCBP}{R_{\mathrm{CBP}}}
\newcommand{\RSD}{R_{\mathrm{SD}}}

\newcommand{\todo}[1]{\textbf{\textcolor{red}{[#1]}}\xspace}
\newcommand{\com}[1]{\textbf{\textcolor{orange}{(#1)}}\xspace}

\usepackage[outputdir={fig},debug]{dot2texi}
\usepackage{draftwatermark}
\usepackage{xspace}
\SetWatermarkText{DRAFT}
\SetWatermarkScale{6}
\SetWatermarkLightness{0.9}
\newcommand{\angexp}{\circ{a}}
\newcommand{\texp}{\ensuremath{\tau}\xspace}
\newcommand{\FixMe}[1]{\textbf{\textcolor{red}{[#1]}}\xspace}
\newcommand{\SZF}[1]{\textbf{\textcolor{blue}{[#1]}}\xspace}
\newcommand{\MARC}[1]{\textbf{\textcolor{green}{[#1]}}\xspace}
\graphicspath{{fig/}}
\makeatletter
\newcommand*\ExpandableInput[1]{\@@input#1 }
\makeatother
\DeclareMathOperator*{\med}{med}
\title{The StarDICE absolute flux calibration experiment: Characterisation of the photometric instrument with a collimated beam projector}

\author{Collaboration Boston Paris}

\abstract{}{}{}{}

\begin{document}

\maketitle

%\section{Currently happening}
%
%\begin{itemize}
%\item SC linearity
%\item pinhole choice
%\item dark current caracterization
%\item Photo-diode change
%\end{itemize}

\tableofcontents

\section{Introduction}


\section{Laboratory setup}

\subsection{StarDice}
\label{sec:stardice}

\todo{Marc} The StarDICE photometric instrument (SPI) consists in a
Newton telescope with a primary mirror of $40$cm diameter (16'') and
$1.6$m focal length ($f/D = 4$). The focal plane hosts an Andor Ikon-M
954 camera, equipped with a thermoelectrically cooled, deep depleted
and back illuminated CCD sensor (E2V DU934P). The active area of the
sensors is $13.3\times13.3 mm$ divided in $1024\times 1024$ square
pixel of $13\mu m$ side. In this baseline setup, the pixel resolution
is $1.68$ arcsec and the field of view $28.6\times28.6$ arcmin.

The 11cm diagonal is oversized to ensure the fully-illuminated plane
extends over the sensor with a confortable margin in all optical
configurations.

[-0.9240625, -13.120000000000001]

\begin{itemize}
\item Filter-CCD distance: max 40.80 mm, min 26.80 mm
\item field of view $30'x30'$
\item Custom built mount
\item filters $ugrizy$
\end{itemize}

\subsection{Collimated Beam Projector}
\label{sec:cbp}

\todo{Jérémy}

The Collimated Beam Projector (CBP) general setup needs the following : one tunable monochromatic lightsource, and an optic device able to recreate a parallel beam from a point source, and mimic a artificial monochromatic light. In our case, we used as a lightsource an EKSPLA NT252 tunable laser, allowing to produce light with a resolution of \SI{1}{\nano\meter}  and a precision about \SI{0.2}{\nano\meter} \com{are the values accurate ?}. The laser produces pulsed light at a frequency of \SI{1}{\kilo\hertz} with two modes. The first one is said "continuous", so it shoots pulses in a continuous way. The second one send packet of photons by "bursts". Each burst is composed from 1 to 1000 pulses, meaning the duration of a burst is restrained between \SI{1}{\milli\second} to \SI{1}{\second}. The light emitted is injected into a device composed of a diaphragm to narrow the light beam, and some filters to purify the light from wavelength contamination. \com{add filters references?}. 

\noindent The light is then injected into an optical fiber of a diameter of \com{??}\SI{}{\micro\meter}, which is plugged into an IS200-4 $\Phi$2'' integrating sphere, composed of 1 input and 4 outputs. Two outputs are plugged to two monitoring instruments. On the first hand we have a Thorlabs SM05PD3A photodiode with FD11A spectral response \com{plot ?}, connected to a Keithley 6514 electrometer to measure the optical power injected into the integrating sphere. On the other hand we have an Ocean QE65000 spectrograph to monitor the wavelength of the light. One output is left free to plug a calibration lamp to calibrate the spectrograph when needed. Finally, a slider with 4 slots is attached to the fourth output. 

\noindent Here, the slider contains 3 pinholes of different size : \SI{75}{\micro\meter}, \SI{2}{\milli\meter} (respectively P75HK and P2000HK from Thorlabs) \SI{5}{\milli\meter} (homemade). The last slot is left empty. The slider is also attached to the ocular of a Ritchey-Chrétien Omegon Pro RC 154/1370 telescope, with a aperture ratio of $f/9$, in order to position the pinhole at the focal point of the optics. The telescope is mounted on a Celestron Nexstar Evolution mount. The figure \ref{fig:sphere} shows a schematic of the integrating sphere and the instrument plugged into it. \\

\begin{figure}[h]
    \centering
    \includegraphics[width=0.7\textwidth]{fig/integrating_sphere.png}
    \caption{Schematic of the integrating sphere}
    \label{fig:sphere}
\end{figure}

The CBP being set up, it has to aim to the instrument we want to shoot into. Hence we have a C60 solar cell of 3$^{\mathrm{rd}}$ generation from Sunpower \com{add ref to sasha's paper}. This solar cell is set on a two axis mount, one that allows a movement in the direction of the optical axis of the CBP telescope, and another axis that is vertical, which allows to adjust the height of the solar cell. This solar cell is connected via a coaxial cable to a Keysight B2987A electrometer, that is able to chop light at a higher frequency than what the Keithley can do, and is a more contemporary instrument. A picture of this setup is shown figure \ref{fig:cbp_setup_sc}. In figure \ref{fig:cbp_setup_sd}, we show a picture of the setup when the CBP is aiming at the StarDice telescope described in section \ref{sec:stardice}.

\begin{figure}
  \begin{subfigure}[b]{0.5\textwidth}
    \includegraphics[width=0.95\linewidth]{fig/cbp_setup_SC.png}
    \caption{CBP setup when shooting in the solar cell.}
    \label{fig:cbp_setup_sc}
  \end{subfigure}
  \hfill %%
  \begin{subfigure}[b]{0.5\textwidth}
    \includegraphics[width=0.95\linewidth]{fig/cbp_setup_SD.png}
    \caption{CBP setup when shooting in the StarDice telescope.}
    \label{fig:cbp_setup_sd}
  \end{subfigure}
\caption{Pictures of the different CBP setups.}
\label{fig:cbp_setup}
\end{figure}


%\begin{itemize}
%\item Ekspla NT252 tunable laser
%\item Light injection system with filters
%\item Optical sphere ref
%\item Ocean QE65000 spectrograph
%\item Photodiode + Keithley 6514. Fig \ref{fig:thorlabs_response}\\
%  \emph{Old one} (used in summer 2021 up to 14th of October 2021): Thorlabs
%  SM05PD1B with FDS100 spectral response data\\
%  \emph{New one} (replaced on the 14th of October 2021): Thorlabs SM05PD3A with
%  FD11A spectral response data

%\item Pinhole slider
%\item Telescope: Ritchey Chretien Omegon telescope\\
%  Apperture ratio $f/9$\\
%  Apperture $154 mm$ \\  
%\item Telescope mount
%\item Solar Cell on movable mount
%\item Keysight B2987A: in order to be able to chop the light at a higher frequency
%  than what the Keithley can do. And it is a more contemporary instrument.
%\end{itemize}
  
%\begin{figure}[!ht]
%    \begin{center}
%      \includegraphics[width=0.8\columnwidth]{thorlabs_response}
%    \end{center}
%    \caption[]{Response curves of the photodiodes. We see the noticeable
%      improvement in the blue for the new one.}
%    \label{fig:thorlabs_response}
%\end{figure}

\subsection{Equations}

\todo{Thierry}

%From equations, draw the data taking plan
%with calibrated wavelength

With the set up described in the section \ref{sec:cbp}, the goal is to obtain the response of the StarDice telescope and its filters transmission. To do so, we will first need to measure the reponse of the CBP optics by shooting into the calibrated solar cell, like shown in figure \ref{fig:cbp_setup_sc}. Once this is done, we can shoot the light coming out of the CBP inside the StarDice telescope to get its reponse, knowing the CBP response. We will need the following quantities : 

\begin{itemize}
    \item $\QPD$: the charges collected in the monitoring photodiode in the integrating sphere, measured by the Keithley 6514 in Coulomb.
    \item $\QSC$: the charges collected in the solar cell and measured by the Keysight 2987A in Coulomb.
    \item $\QSD$: the charges collected by the Andor camera of the StarDice telescope in ADU \com{???}.
\end{itemize}

\noindent Thanks to the quantities $\QPD$ and $\QSC$, and knowing the quantum efficiency of the solar cell $\epsilon_{\mathrm{SC}}$ from \com{sasha's paper}, as well as the elemental charge of the electron $e$, we can infer obtain the response of the CBP optics $\RCBP$ thanks to the formula \ref{eq:rcbp}.  Hence, it is possible to infer the equation for $\RSD$ knowing $\RCBP$, according to the equation \ref{eq:rsd}. 

\begin{equation}
    \RCBP = \frac{\QSC}{\QPD \times \epsilon_{\mathrm{SC}} \times e}
    \label{eq:rcbp}
\end{equation} 

\begin{equation}
    \RSD = \frac{\QSD}{\QPD \times \RCBP}
    \label{eq:rsd}
\end{equation}


\noindent These two equations will be the most important ones in the following, and we will focus on $\RCBP$ and $\RSD$ respectively on the sections \ref{sec:rcbp} and \ref{sec:rsd}.

\subsection{Strategy}

The goal can be decompose in 4 
Final measurement shcedule :
\begin{itemize}
    \item Measure CBP response for [350-1100] nm range : Shoot in Solar Cell with a 5mm pinhole
    \item Measure StarDice response for [350-1100] nm range : Shoot at a given position in StarDice for every filter with 75µm, 2mm and 5mm pinhole
    \item Measure the uniformity of the StarDice mirror (“pupil stitching”) : Shoot in StarDice at 8 different positions on the mirror with 5mm pinhole
    \item Measure the uniformity of the StarDice focal plane : Shoot in StarDice telescope at 16 different positions on the focal plane (CCD)
\end{itemize}

Detailed schedule for final measurement :
\begin{itemize}
    \item Spectrograph calibration
    \item 8 positions on the mirror; 75µm
    \item 4 positions on different quadrants, no filter, but same radius 
    \item 4 positions at different radius but same quadrant, every filters
    \item 3 repeatability measurements : same position ; every filter ; 75µm
    \item 5 SolarCell runs : QSW = 298, MAX ; 5mm
    \item 5 StarDice runs : EMPTY filter ; 5mm
    \item 5 SolarCell runs : QSW = 298, MAX ; 5mm
    \item 1 StarDice run : pinholes 75µm, 2mm, 5mm
    \item 2 SolarCell runs + 1 StarDice run : cap on the CBP to see ambient light
    \item 2 SolarCell runs : long and short distance (~16cm difference) ; 5mm
    \item (4x4) positions on the CCD ; no filter ; 75µm
    \item Spectrograph calibration
\end{itemize}








\section{CBP response calibration with a solar cell}
\label{sec:rcbp}

\todo{Jérémy}

Foundations: The Solar Cell QE (Sasha and Elana et al.)


\subsection{Data set description}


\subsubsection{Choice of the dynamic range}
We need to select the number of pulses per wavelength. We decide to keep the
photo-diode current as flat as possible in order to keep the non-linearity of
the Keithley 6514 (that we suppose the largest non-linear component of our
setup) as small as possible.


\begin{figure}[!ht]
\begin{center}
\includegraphics[width=0.8\columnwidth]{calculate_npulses}
\end{center}
\caption[]{Number of pulses per bursts needed to get a constant signal from the photodiode.}
\label{fig:calculate_npulses}
\end{figure}

This was for the Thorlabs \Oldthorlabs


\subsubsection{Dark current characterization}

Measurement and selection of a dark current that doesn't saturate the Keisight
too fast. Measurement of the dark current noise level. 

\begin{itemize}
\item Without offset box: $\sim 20 nA$
\item With offset box: $~ <0> nA \pm 5 nA$
\end{itemize}


Dark current with laser off over 45 minutes, mimicking 10 second observation
blocks corresponding to the 5 laser bursts.
\begin{figure}[!ht]
\begin{center}
%\includegraphics[width=0.8\columnwidth]{dark_current_spectrum.jpeg}
\end{center}
\caption[]{Left: signal auto-correlation. Right: signal FFT. Dominated by the
  stitching of 10 second blocs together.}
\label{fig:darkcurrentspectrum}
\end{figure}

Connecting the Keysight to a $\SI{243}{\ohm}$ resistor gives exactly the same
correlation function. The dark current comes entirely from the Keysight


\subsubsection{Current or charge mode}

\subsubsection{Choice of the pinhole}

\subsubsection{Masking of the beam}
We us a $3/4$ pie-slice mask at the end of the CBP telescope in order to reduce
the beam size, which would otherwile overfill the Solar Cell surface.

Checks to demonstrate that this doesn't impact, via inner reflexions, the
calibration of the CBP telescope.

\subsubsection{Focussing and distance: last focussing 2021-10-11}
The CBP was aligned with the StarDICE telescope to improve the focus. The
StarDICE camera was set close to infinity with no filter (focus encoder:
8mm). The smallest pinhole (75microns). Adjustment of the CBP focus was
performed by Elana in order to get the smallest extension of the pinhole
image. A slight offset was then added to account for the small change in focus
introduce by tightening the locking screws. Here is a confirming image taken
just after.

The focus was check to resist slot changes and even complete removing and
reinstallation of the pinhole.

For the record mount coordinates to get CBP and telescope aligned were
$[13.199987411499023, 1.4999985694885254]$

The measurements at 2 different distances (separated by $16 cm$) show a 5\%
difference in flux both for the 5mm and the 2mm pinhole.

\begin{figure}[!ht]
\begin{center}
%\includegraphics[width=0.8\columnwidth]{distance_variation.png}
\end{center}
\caption[]{The data was taken with the integrating sphere port open. We should
  replace this by more recent data/analysis.}
\label{fig:distancevariation}
\end{figure}


\subsubsection{Available pinholes}

Choice of a pinhole that allows enough S/N. Since we have to switch pinholes,
verification of the relation between flux and pinhole size. Verification of
pinhole achromaticity.

\begin{itemize}
\item Various pinhole sizes measurements with Solar Cell
\item Various pinhole sizes measurement with telescope
\end{itemize}



\subsubsection{Light source filtering}

We suspect that the laser produces harmonics that need to be filtered
out. Measurement of those harmonics and selection of filters to filter them
out. 


\subsubsection{Spectrograph}


\subsection{Data reduction}



\subsubsection{Photodiode data reduction overview}

\subsubsection{Solar Cell data reduction overview}

\subsubsection{Spectro data reduction}


\subsection{CBP response}

\subsection{Systematics}


\subsubsection{External light contamination}

Varying QSW we assume that seen non-linearities come from an external light component

\subsubsection{Varying the SC distance and angle}

to do ???


\subsubsection{Solar Cell linearity}

The Solar Cell has small shunt resistance and we light it with a powerful source
of very short pulse length. Verification of linearity w.r.t flux and w.r.t
number of pulses per burst. 


\begin{itemize}
  \item QSW sequence  (Fig. \ref{fig:SCqswlinearity})
  \item Behavior with change of number of pulses
  \item Behavior of solar cell at different distances
\end{itemize}


\begin{figure}[!ht]
\begin{center}
\includegraphics[width=0.8\columnwidth]{solar_cell_qsw_linearity.jpg}
\end{center}
\caption[]{Something}
\label{fig:SCqswlinearity}
\end{figure}


\section{Stardice response}
\label{sec:rsd}

\subsection{Data set description}
\label{sec:datadesc}

\begin{itemize}
\item Images
\item Photocurrent timeseries
\item Spectral time series
\end{itemize}

\subsection{Reduction of images}
\label{sec:photometry}

\todo{phrase pour dire que c'est pareil pour photodiode et spectro}

\subsubsection{5mm pinhole}

+ courbe de croissance

\subsubsection{75um pinhole}

\subsubsection{Ghost photometry}

\subsection{5mm, 75um pinhole intercalibration}

Link SC with StarDICE :
- rapport 5mm/75um

\subsection{Stardice response scan}

\subsubsection{CCD grid}

\subsubsection{Mirror scan}

\subsubsection{Radius scan}

\subsubsection{Ghost buster and beam impact}

\subsubsection{On sky pupil synthesis and filters}



\subsection{Systematic uncertainties}
\label{sec:systematics}

\subsubsection{Stability of the StarDice responses}

\subsubsection{Gain and linearity}
\label{sec:gain}

Varying pinhole with StarDice, and CBP
Varying QSW but depending on result it falls into this subsection or the following

\subsubsection{Pinhole chromaticity}

\subsubsection{Pull distributions}

\subsubsection{Courbes de croissances}


\section{Discussion}

\subsection{Instrument model}\label{sec:model}

\todo{Marc: mais plus tard après analyse, et/ou après tuto}

\todo{Direct transmission measurement, ghost brightness model, angle dependence,  Pupil synthesis}

\section{Conclusion}
\label{sec:conclusion}




\end{document}

